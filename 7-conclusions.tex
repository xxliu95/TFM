\chapter{Conclusions and Future Work}
\label{chap:conclusions}

\section{Conclusions}

This Master Thesis have the objectives of building a simulation framework
able to test \textit{ABR} adaptation algorithms in mobile network scenarios,
and comparing different implmentations of existing adaptation algorithms 
based on QoE and fairness metrics.

First, the study begin by familiarizing with the simulation software \textit{ns-3}.
It was really challenging at first, due to the lack of grafical user interface
and relatively small community to help solving problems. Learnt scripting
and implementing new modules in \textit{ns-3}.

Also, become familiar the \textit{LTE} technologies. Including the architecture 
of the \textit{EPC}, radio interface, wireless fundamentals and protocol stack layers.
The \textit{LTE} module in \textit{ns-3} made possible the simulation of \textit{ns-3}
applications in mobile scenarios.

To achieve the objectives proposed, the design of a new \textit{ns-3} module is developed.
And also the implementation and integration with other \textit{ns-3} modules. The implemented 
\textit{ABR} module mimics real-world video players behaviour and
can work as a framework to test new adaptation algorithms, and extract
metrics of the simulation. The most difficult part was the implementation of the 
\textit{BOLA} algorithm, it was really complex and hard to understand. An additional
feature was discarted, but can be done in futere work is to parse MPD files as a
input for the simulation.

Finally, various simulations are made to compare \textit{QoE} and fairness metric between
the implemented \textit{ABR} adaptation algorithms. This new module has its limitations
and can be improved with future work. 



\section{Future Work}

These are possible improvements for future work:

\begin{itemize}
  \item \textbf{Implement more adaptation algorithms}. To be able to compared more existing
  adaptation algorithms or even put to test new designs of algorithms.
  \item \textbf{Traffic shapping}. To analyse the effect of applying traffic shapping techniques 
  like token bucket and leaky bucket.
  \item \textbf{TCP congestion control}. As commented on the thesis, the \textit{TCP} congestion 
  control influents the adaptation algorithms. It could be interesting testing scenarios.
  \item \textbf{X2 Handover}. The scripts implemented is limited to one \textit{eNodeB}. There 
  is a possibility in \textit{ns-3} to use multiple \textit{eNodeB} and handover over the \textit{X2}
  interface.
  \item \textbf{5G}. There is a new \textit{5G} module in \textit{ns-3} although in its very 
  early stages. \textit{5G} simulation scenarios could be also worth to look at.
\end{itemize}
