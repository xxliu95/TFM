\chapter{State of the Art}
\label{chap:soa}

This chapter will introduce the main concepts and tools that will be used during the development 
of the project. The \autoref{sec:abr} will explain the different methods of content distribution 
over \textit{HTTP} and different types and implementations of adaptive streaming.
The \autoref{sec:dash} will make a introduction to 
the \textit{DASH} standard, different types of adaptation algorithms and \textit{QoE} and 
\textit{QoS} metrics. The \autoref{sec:4g} will describe basic architecture and fundamentals
of 4G LTE, such as the radio interface, propagation loss model, fading model, antenna model, etc.

\section{ABR Video Streaming}
\label{sec:abr}

There are three ways of media delivery over \textit{HTTP}. The first method is by
\textbf{file download}, the media file is downloaded in its entirety in a local hard disk and 
then it can be played. The second method is called \textbf{progressive download}, this method
is similar to the file download, but instead the download starts from the beginning and the 
media starts playing once enough data are playable. 
However, these two methods have disadvantages like waste of bandwidth or
\textit{DRM} issues and also requiring a reliable transmission. The last method is called
\textbf{streaming}, contrary to the former two, the file itseft is not stored locally, 
smaller chunks of video are sent from the server and the client needs a data buffer to store 
the data that is being downloaded. The client plays the multimedia content from the 
buffer, and when the session is closed the data are deleted.

Streaming media also comes with some challenges. There are a lot of network variability
and a big heterogeneity in video capable devices. Therefore, to overcome these shortcomings,
\textit{Adaptive bitrate streaming (ABR)} was created.

The basic idea of \textit{Adaptive bitrate streaming} is to adapt the media content
for the user by monitoring different parameters like estimated bandwidth, buffer level or
\textit{CPU load}, see \autoref{fig:abrtime}. There are many propietary adaptive streaming solutions:

\begin{itemize}[topsep=0.5pt]
  \setlength\itemsep{0.5pt}
  \item \textbf{\textit{Apple HTTP Live Streaming (HLS)}}: \textit{HTTP Live Streaming HLS}
  is an implementation of an \textit{ABR} protocol over \textit{HTTP} developed by Apple \cite{hls1}
  as part of the QuickTime software and the mobile operating system \textit{iOS}. \textit{HLS} 
  supports live streaming and video on demand. \textit{HLS} is proposed in 2009 as a standard to
  the \textit{IETF} \cite{hls2}.
  \item \textbf{\textit{Microsoft Smooth Streaming (MSS)}}: \textit{Smooth Streaming} is part
  of \textit{Internet Information Services (IIS) Media Services} for delivering media over
  \textit{HTTP} \cite{mss1}. Their \textit{MSS} technology was used for several
  sports events such a the Beijing Summer Olympic Games in 2008 and the 2010 Winter Olymplics
  in Vancouver \cite{mss2}.
  \item \textbf{\textit{Adobe HTTP Dynamic Streaming (HDS)}}: \textit{HTTP Dynamic Streaming}
  is the implementation of adaptive streaming by Adobe. \textit{HDS} enables high-quality, network
  efficient HTTP streaming for media delivery that is tightly integrated with Adobe software \cite{hds1}. The
  solution is based in using \textit{Open Source Media Framework (OSMF)} and Adobe Flash Player.
\end{itemize}

\begin{figure}[h]
  \centering
  \includegraphics[width=0.7\textwidth]{img/abrtime.png}
  \caption{Evolution of segment quality with time}
  \label{fig:abrtime}

\end{figure}

But there was no official standarization for adaptive video delivery over HTTP. For that reason,
a new international stadard called \textit{MPEG-DASH} was developed and published.

\section{Dynamic Adaptive Streaming over HTTP}
\label{sec:dash}

\textit{DASH} was developed from January 2009 to March 2010 and published in April 2012. 
The most recent revision of the standarization was released in 2019 as 
\textit{MPEG-DASH ISO/IEC 23009-1:2019} \cite{ISO23009}. 
 \textit{Moving Picture Experts Group} from \textit{ISO/IEC} and the \textit{3GPP} collaborated on
the \textit{DASH} standard. The \textit{3\textsuperscript{rd} Generation Partnership Project} defined the use of
\textit{DASH} as the standard of digital media delivery in mobile networks (4G \textit{LTE}, 5G) in \cite{3gpp1}.

The objective of \textit{DASH} was to create a unique standard that replaces the propietary solutions
from Microsoft, Apple and Adobe. Also, it will offer the interoperability and the convergence needed for 
the expansion of large-scale video streaming solutions. Also, the \textit{DASH Industry Forum (DASH-IF)} was created to promote and help the expansion of
\textit{DASH}. Microsoft, Apple, Netflix, Qualcomm, Ericsson and Samsung are some of the companies
members of the \textit{DASH-IF}.

One of the biggest advantages of \textit{DASH} is that the video streaming is over \textit{HTTP} version 1.1 protocol
(\textit{HTTP/1.1}). The use of \textit{HTTP} means that reusing existing internet infrastructure and
media content distribution tecniques using \textit{CDN (Content Delivery Networks)} can be done.
Another convenience of using \textit{DASH} is that due to using \textit{HTTP} encapsulation, problems
with passing through firewalls and the \textit{Network Address Translation (NAT)}
are not existent.

All the control of the media content delivery is located in the \textit{DASH} client side. The
standard does not define any web delivery mechanism nor the bitrate adaptation algorithm. What \textit{DASH}
does define in \cite{ISO23009} are:

\begin{itemize}
  \item \textit{\textbf{The Media Presentation Description (MPD) File Format}}: The \textit{MPD} file
  uses the \textit{eXtensible Markup Language (XML)} and
  contains the specifications of the media content and the \textit{URL} of the segments
  in the \textit{HTTP} video servers.
  \item \textbf{Segment format}: \textit{DASH} defines the characteristics of the necessary
  codifications and the way that the media content is divided in small fragments called 
  \textit{segments}.
\end{itemize}


\begin{figure}[h]
  \centering
  \includegraphics[width=0.7\textwidth]{img/dasharch.png}
  \caption{DASH client-server architecture. Source: MPEG \cite{ios1}}
  \label{fig:dasharch}
\end{figure}

The \autoref{fig:dasharch} presents a simple \textit{DASH} architecture. The video and audio
content are processed and stored on an \textit{HTTP} server. To access the content, the client
sends \textit{HTTP} requests to the server. But first, the client needs to download the 
\textit{MPD} file, normally through \textit{HTTP}. The client then does the
parsing of the \textit{MPD}, extract information such as the duration of a segment, media types or 
resolutions. Finally, the \textit{DASH} client chooses the adequate quality and starts the 
streaming of the content using \textit{HTTP GET} request to fetch the segments.

The \textit{DASH} client stores the segments in a buffer and consumes the content. It continues
to fetch new segments and by monitoring network variables it will decide which quality (higher
or lower bitrate) to request next to avoid problems like buffer underflow and maintain at 
least a set number of segments in the buffer.

\subsection{MPD}
\label{sec:mpd}
The \textit{MPD} file is an \textit{XML} document that describes the characteristics
of the different media components that composes the media content (e.g. video, audio, subtitles).

The structure of the \textit{MPD} is hierarchical as illustrated in \autoref{fig:mpd}. The media content is divided in a sequence of
\textbf{periods}, each period has a starting time and a duration. In a period, the set of encoded
versions of the media content is consistent, that is, the same bitrates, languages and so on.

\begin{figure}[h]
  \centering
  \includegraphics[width=\textwidth]{img/mpd.png}
  \caption{The MPD hierarchical data model. Source: MPEG \cite{ios1}}
  \label{fig:mpd}
\end{figure}

Each period consists of one or multiple \textbf{adaptation sets}. A collection of interchangeable 
encoded versions of one or more media content components is referred to as an adaptation set. For instance,
and adaptation set may contain the different bitrates of the video component of the same multimedia content
and another adaptation set may contain the different bitrates of the audio component of the same multimedia
content.

An adaptation set contains a set of \textbf{representations}. A representation describe an enconded
alternative of the same media component, the alternatives can vary by bitrate, resolution, framerates, 
codec, sampling rate or other characteristics.

Each representation consists of one or multiple \textbf{segments}. A segment is the media stream chunks
in temporal sequence. Each segment has a \textit{URI}, the client will use this \textit{URI} to make
\textit{HTTP GET} requests to the video server. 


\subsection{Adaptation Algorithms}
\label{sec:adap}

In a video streaming service, there are a number of factors like the download bandwidth, delay or packet losses
that can produced undesirable effects on the client such as buffer underflow, rebuffering and interruptions
that lead to bad playback experience, thus, a bad Quality of Experience. To solve these problems, the ABR video
streaming clients uses different adaptation algorithms to give a higher QoE.

An adaptation algorithm is a technique used in a multimedia streaming service to adjust the video quality
in real-time according to different parameters. Some of the parameters are:

\begin{itemize}[noitemsep,topsep=0pt]
  \item \textbf{Client device}: The screen resolution, CPU capabilities, Buffer size, etc.
  \item \textbf{Network}: Type of access network (Mobile, Fixed), available bandwidth, etc.
\end{itemize}

The following subsections will explain different types of adaptation algorithms and the algorithms implemented
for this thesis in \textit{ns-3}.

\subsubsection{Bandwidth throughput based algorithms}

This group of algorithms uses estimations of bandwidth throughput as the main rule to select
the qualities of the multimedia content for the client. The main difference between algorithms of
this kind is the bandwidth estimation method and how the estimation relates to the qualities. 

\begin{figure}[h]
  \centering
  \includegraphics[width=\textwidth]{img/dashjs.png}
  \caption{Bandwidth based algorithms. Source: \cite{abr1}}
  \label{fig:throughput}
\end{figure}

\begin{itemize}
  \item \textbf{HLS.js} \cite{hls3}. The algorithm is called Bandwidth estimation. 
  
  The algorithm processes two EWMA (Exponentially Weighted Moving Averages) and chooses the minimum of the two 
  as the bandwidth estimation.
  Then the bandwidth estimation is multiplied by a factor to reduce oscilation. And finally it selects the 
  first quality with a bitrate less than the adjusted bandwidth estimation. 


  \item \textbf{DASH.js} \cite{dash3}. The Throughput Rule.
  
  This algorithm is basically the same as the Bandwidth estimation from HLS.js.
  It computes the average throughput, and uses an safety factor to avoid oscilations. And then chooses the quality
  based on the safe average and creates a new \textit{SwitchRequest}.

  
\end{itemize}

\subsubsection{Buffer based algorithms}

This group of algorithms uses buffer occupancy information to try to choose the highest level of bitrate
for the multimedia content. These algorithms are usually used to avoid buffer underflow.

\begin{itemize}
  \item \textbf{BOLA} \cite{bola1}. Buffer Occupancy based Lyapunov Algorithm.
  
  The BOLA adaptation algorithm uses the Lyapunov optimization to make decisions. This is an utility 
  theory and it is configurable with a tradeoff parameter to choose between rebuffering potential and bitrate
  maximization.
  
  \begin{figure}[h]
    \centering
    \includegraphics[width=0.75\textwidth]{img/BOLA.png}
    \caption{BOLA's bitrate choice as function of buffer level. Source: \cite{bola1}}
    \label{fig:bola}
  \end{figure}

  BOLA tries to maximize ${V_{n}+\gamma S_{n}}$.
  where: 
  \begin{itemize}
    \item[$\circ$] \textbf{${V_{n}}$} is the bitrate utility.
    \item[$\circ$] \textbf{${S_{n}}$} is the playback smoothness.
    \item[$\circ$] \textbf{${\gamma}$} is the tradeoff weight parameter.
  \end{itemize}
\end{itemize}


\subsubsection{Control theory based or hybrid algorithms}

This class of algorithms uses a combination of throughput estimation and buffer occupancy and tries to 
maximize the bitrate selection with decision-taking indicators calculated making use of control theory or
stochastic optimal control equations.

\subsection{QoS \& QoE Metrics}
\label{sec:qoemetrics}


The \textit{Quality of Service (QoS)} is defined by the \textit{ITU-T} in the document P.10/G.100
 \cite{itu2} as \textquotedbl The totality of characteristics of a telecommunications service that bear on its 
 ability to satisfy stated and implied needs of the user of the service\textquotedbl. And the \textit{Quality of 
Experience (QoE)} is defined as \textquotedbl The degree of delight or annoyance of the user of an application or service\textquotedbl.

The standard \textit{ISO/IEC 23009} defines a list of parameters for \textit{Quality of Service (QoS)} and
\textit{Quality of Experience (QoE)} for the adaptation algorithms to base on. There parameters 
is also used to evaluate the overall quality in the multimedia distribution service.

Some of the metrics defined in \cite{3gpp1} and \cite{ISO23009} are as follows:

\begin{itemize}
  \item \textbf{Average Throughput}: This is a \textit{QoE} metric that defines a list in which 
  the average Throughput observed in the client during a measuring period.
  \item \textbf{Initial Playout Delay}: This is a \textit{QoE} metric that represents the initial 
  delay in the reproduction of the media content.
  \item \textbf{Representation Switch Events}: This is a \textit{QoS} metric for measuring the 
  number of representation switch events of the multimedia content.
  \item \textbf{Buffer Level}: This is a \textit{QoS} metric that monitors the level of occupancy
  of the buffer during the reproduction of the multimedia content.
\end{itemize}


\section{LTE}
\label{sec:4g}

\textit{Long Term Evolution (LTE)} was first introduced in 2008 in the Release 8 of the \textit{3GPP}
specification \cite{lte1}. The objective of \textit{LTE} was to migrate the \textit{3GPP} systems
into a optimized system based on packet switching (all \textit{IP}), with greater bitrates, lower
latency y multiple radio access technologies support.

\subsection{History}
\label{sec:4gintro}

The first mobile phone call was made in 1973 \cite{mob1}. New generations of mobile networks 
are developed almost every decade. The first generation 1G launched years later, but
it was only capable of doing voice calls. In 1991, the second generation 2G \textit{(GSM)} of 
mobile networks was introduced. \textit{GSM} provided improved wireless capabilities and 
introduced by the first time multimedia content with \textit{Multimedia Message Service (MMS)}.
But it was the third generation 3G, launched in 2001, that enabled new internet-driven
services such as video conferencing and streaming. Later in 2009, the \textit{LTE} 4G standard
was commercially deployed. With theorical download bandwidth of almost 100Mbps made high-quality
streaming into reality. 5G technologies improves in bandwidth even more and brings 
video streaming in \textit{UHD} and more.

The consumption of multimedia content on mobile networks is becoming increasingly relevant with 
the rise of bandwidth and ease of access. This section will provide a brief introduction to the 
basic concepts of mobile networks, their architecture and fundamentals.

\subsection{Architecture}
\label{sec:eps}

The design of the \textit{LTE} architecture was done from the ground up. The goal was to build a flat, all
\textit{IP} architecture using packet-switching, well structured (separation of control plane and user plane)
and with few elements.

\begin{figure}[h]
  \centering
  \includegraphics[width=0.8\textwidth]{img/eps.png}
  \caption{LTE Architecture}
  \label{fig:eps}
\end{figure}

The \textit{Evolved Packet System (EPS)} is constituted by the following elements:

\begin{itemize}
  \item \textbf{\textit{User Equipment (UE)}}: An \textit{UE} is any device used by an end user
  to communicate in a mobile network.
  \item \textbf{\textit{Evolved UMTS Terrestial Radio Access Network (E-UTRAN)}}: The only elements
  in the \textit{E-UTRAN} are the \textit{e-NodeB}. An \textit{\textbf{enhanced Node B (e-NodeB)}} works as a base station
  and a controller.
  \item \textbf{\textit{Evolved Packet Core (EPC)}}: The \textit{EPC} is made up of a network of gateways, 
  control servers, and databases linked by a \textit{IP} backbone. The main elements of the \textit{EPC}
  are:
  \begin{itemize}
    \item[$\circ$] \textit{\textbf{Mobility Management Entity (MME)}}: The \textit{MME} is a server 
    used for managing the signalling of the operation. 
    \item[$\circ$] \textit{\textbf{Serving Gateway (SGW)}}: The \textit{SGW} is the gateway used for
    communicating the access network \textit{E-UTRAN} and the \textit{PGW}.
    \item[$\circ$] \textit{\textbf{Packet Data Network Gateway (PGW)}}: The \textit{PGW} is the gateway
    for the traffic between the core network and external packet data networks. 
    \item[$\circ$] \textit{\textbf{Home Subscriber Server (HSS)}}: The \textit{HSS} is a database 
    containing information about the \textit{EPC} network users.  
    \item[$\circ$] \textit{\textbf{Policy Charging and Rule Function (PCRF)}}: The \textit{PCRF} is used
    for \textit{QoS}, policy and charging management. 
  \end{itemize}
\end{itemize}

\begin{figure}[h]
  \centering
  \includegraphics[width=0.8\textwidth]{img/epc.png}
  \caption{Evolved Packet Core (EPC) Architecture}
  \label{fig:epc}
\end{figure}

\subsection{OFDMA and SC-FDMA}
The cellular communication systems needs to have a strategy for multiple access. In LTE, the 
\textit{Orthogonal Frequency Division Multiple Access (OFDMA)} is used for downlink and the \textit{Single-
Carrier Frequency Division Multiple Access (SC-FDMA)} is used for uplink. Both are very similar, consisting
in allocating each subscriber some portion of the subcarriers for certain amount of time.

In the Figure \ref{fig:lterb}, a transmission structure of LTE is presented. The two dimentions of the 
plane are time and frequency. Two important concepts are defined as:

\begin{figure}[h]
  \centering
  \includegraphics[width=0.95\textwidth]{img/lte_rb.png}
  \caption{LTE Time-Frequency Grid. Source:\cite{cmov1} }
  \label{fig:lterb}
\end{figure}

\begin{itemize}
  \item \textbf{\textit{Resource Element (RE)}}: A Resource Element is the basic element of resouce, it is
  defined as one subcarrier in a symbol period.
  \item \textbf{\textit{Resource Block (RB)}}: A Resource Block is composed by twelve subcarriers (180 kHz) in 
  a time interval of 0.5 ms (7 OFDM symbols).
\end{itemize}


Users are assigned resources in resource blocks across a subframe, i.e., 12 subcarriers over ${2\times7 = 14}$
OFDM symbols for a total of 168 Resource Elements. Because some of the 168 resource components are utilized 
for various layer 1 and layer 2 control messages, not all of them can be used for data.

The number of Resource Blocks available for each channel bandwidth is given by the Table \ref{table:rb}. 

\begin{table}[h]
  \centering
  \begin{tabular}{@{}lcccccc@{}}
  \toprule
  \textbf{Bandwidth}               & 1.4 MHz & 3 MHz & 5 MHz & 10 MHz & 15 MHz & 20 MHz \\ \midrule
  \textbf{Number of RBs available} & 6       & 15    & 25    & 50     & 75     & 100    \\ \bottomrule
  \end{tabular}
  \caption{Number of Resource Blocks against each channel bandwidth. Source: \cite{ofdma1}}
  \label{table:rb}
\end{table}


\subsection{Wireless Fundamentals}

Large-scale wireless networks, such as LTE, are fundamentally inefficient and prone to 
interference. Supporting mobility while also obtaining high levels of power efficiency, 
such as through directional antennas, can be really challenging. Base stations must be 
selectively installed but accommodate vast user populations in order to be cost-effective, 
resulting in a significant amount of self-interference. As a result, achieving high coverage, 
capacity, and dependability at low cost and used power is extremely difficult, if not impossible.

The following list highlights the main parameters affecting the received signal in a wireless system. 


\subsubsection{Propagation loss} 

The amount of transmitted power that actually reaches the receiver is the first visible 
difference between wired and wireless channels. The transmitted signal energy extends along 
a spherical wavefront if an isotropic antenna is utilized, hence the energy received at an 
antenna ${d}$ distant is inversely proportional to the sphere surface area, ${4\pi d^2}$.
However, in reality the propagation environment is not free space, we may also take into
account other factors such as reflections.

\subsubsection{Shadowing} 

Obstacles such as trees and buildings, as shown in Figure \ref{fig:pathloss}, may be situated between the 
transmitter and receiver, causing temporary signal degradation, whereas a temporary line-of-sight 
transmission path would result in abnormally high received power.

\begin{figure}[]
  \centering
  \includegraphics[width=0.77\textwidth]{img/pathloss.png}
  \caption{Shadowing effect. Source:\cite{lte2} }
  \label{fig:pathloss}
\end{figure}

\subsubsection{Fading loss} 

The fading effect is another aspect of wireless channels. Fading is generated by the receiving 
of multiple versions of the same signal (multipath), unlike path loss or shadowing, which are large-scale 
attenuation effects induced by distance or obstacles.

The reflections may arrive at very short intervals. For example, if there is local dispersion 
around the receiver, or they may arrive at relatively longer intervals, for instance, if the 
transmitter and receiver are on multiple pathways. Figure \ref{fig:fading} illustrates this.

\begin{figure}[]
  \centering
  \includegraphics[width=0.77\textwidth]{img/fading.png}
  \caption{Fading loss effect. Source:\cite{lte2} }
  \label{fig:fading}
\end{figure}

\subsection{Antennas \& MIMO}

An antenna is a device that uses electromagnetic waves to transmit or receive information.
The transmitting antenna turns electrical currents into electromagnetic waves, 
and vice versa (receiving antenna).

\textit{Multiple Input, Multiple Output (MIMO)} is a technique for increasing the capacity 
of a radio link by employing multiple transmitting and receiving antennas to take advantage
of multipath propagation. MIMO has become a key component of wireless communication technologies 
such as LTE.

There are also special cases of MIMO:
\begin{itemize}[topsep=0pt]
  \item \textbf{\textit{Multiple-input single-output (MISO)}}: When there are multiple transmitting antennas and a single antenna.
  \item \textbf{\textit{Single-input multiple-output (SIMO)}}: When the transmitter has a single antenna and there are multiple receiving antennas.
  \item \textbf{\textit{Single-input single-output (SISO)}}: SISO is a radio system in which neither the transmitter nor the receiver has multiple antennas.
\end{itemize}

Another special type of MIMO is called \textit{Multi User-Multiple Input Multiple Output}.
Single-user SU-higher MIMO's per-user throughput is better suited to more sophisticated user devices with more antennas, 
whereas MU-MIMO is more practical for low-complexity mobile phones with a small number of reception antennas.

\subsection{Physical Layer}

LTE enb phy

UM buffer size
Earfcn

