\chapter{ns-3} 
\label{chap:ns3tuto}

\section{Getting Started}

The prerequisites for the \textit{ns-3} release version 3.32 are the following tools:

\begin{table}[h]
  \centering
  \begin{tabular}{@{}ll@{}}
  \toprule
  Prerequisite & Package/version                             \\ \midrule 
  C++ compiler & \texttt{clang++} or \texttt{g++} (\texttt{g++} version 4.9 or greater) \\
  Python       & \texttt{python3}  version \textgreater{}= 3.5         \\
  Git          & any recent version                          \\ 
  tar          & any recent version                          \\ 
  bunzip2      & any recent version                          \\ \bottomrule \\
  \end{tabular}
  \caption{Prerequisites for ns-3}
\end{table}

Start by downloading the source archive from \href{https://www.nsnam.org/release/ns-allinone-3.32.tar.bz2}{nsnam}
or \href{https://gitlab.com/nsnam/ns-3-allinone.git}{gitlab}. Then build \textit{ns-3} with \texttt{build.py}:

\begin{lstlisting}[language=myshell,caption={Download and installation of ns-3}, captionpos=b]
  # Download from nsnam
  $ cd
  $ mkdir workspace
  $ cd workspace
  $ wget https://www.nsnam.org/release/ns-allinone-3.32.tar.bz2
  $ |\color{myblue}tar| xjf ns-allinone-3.32.tar.bz2
  $ cd ns-allinone-3.32
  # Building ns-3
  $ ./build.py --enable-examples --enable-tests
  # Running a script
  # Create or copy a script to the scratch directory
  $ cp examples/tutorial/first.cc scratch/myfirst.cc
  $ ./waf --run scratch/myfirst
\end{lstlisting}

\textbf{Logging Module}

Enable logging:

\begin{lstlisting}[escapechar=@, language=myshell,caption={Enabling logging in ns-3}, captionpos=b]
  $ export 'NS_LOG=UdpEchoClientApplication=level_all|prefix_func|prefix_time:UdpEchoServerApplication=level_all|prefix_func|prefix_time'
\end{lstlisting}

To disable logging simply type:

\begin{lstlisting}[escapechar=@, language=myshell,caption={Disabling logging in ns-3}, captionpos=b]
  $ export NS_LOG=
\end{lstlisting}

\textbf{Command Line Arguments}

An example of a command could be like this:

\begin{lstlisting}[escapechar=@, language=myshell,caption={Command line arguments}, captionpos=b]
  # To check help 
  $ ./waf --run "scratch/myfirst --PrintHelp"
  # To check variables for PointToPointNetDevice
  $ ./waf --run "scratch/myfirst --PrintAttributes=ns3::PointToPointNetDevice"
  # We set the Datarate to 5Mbps
  $ ./waf --run "scratch/myfirst --ns3::PointToPointNetDevice::DataRate=5Mbps"
\end{lstlisting}

\textbf{ASCII Tracing}

To enable ASCII Tracing, right before the call to \texttt{Simulator::Run ()}, add the following lines
of code:

\begin{lstlisting}[escapechar=@, language=myC++,caption={ASCII tracing}, captionpos=b]
  AsciiTraceHelper ascii;
  pointToPoint.EnableAsciiAll (ascii.CreateFileStream ("out.tr"));
\end{lstlisting}

This will generate the output from \texttt{pointToPoint} to a file named \texttt{out.tr}.

\textbf{PCAP Tracing}

To enable \textit{pcap} tracing simply add:
\begin{lstlisting}[escapechar=@, language=myC++,caption={PCAP tracing}, captionpos=b]
  pointToPoint.EnablePcapAll ("myfirst");
\end{lstlisting}

\textbf{Sockets}

Creating a socket:

\begin{lstlisting}[language=myC++, caption={ns-3 Socket programming}, captionpos=b]
  Ptr<Node> node;
  // Create a TCP socket
  Ptr<Socket> mySocket = Socket::CreateSocket (node, TcpSocketFactory::GetTypeId ());
  // Functions
  mySocket->Bind();
  mySocket->Connect( ... );
  mySocket->Send( ... );
  mySocket->Recv( ... );
  mySocket->Close();
\end{lstlisting}

For callbacks:

\begin{lstlisting}[language=myC++, caption={Socket callbacks}, captionpos=b]
  mySocket->SetConnectCallback (
    MakeCallback (&MyClass::ConnectionSucceeded, this),
    MakeCallback (&MyClass::ConnectionFailed, this)
  );
  mySocket->SetSendCallback (MakeCallback (
    &MyClass::HandleSend, this));
  mySocket->SetRecvCallback (MakeCallback (
    &MyClass::HandleRead, this));
\end{lstlisting}

\textbf{PointToPoint NetDevice}

\begin{lstlisting}[language=myC++, caption={PointToPointHelper}, captionpos=b]
  NodeContainer n;
  n.Create (2);
  PointToPointHelper p2ph;
  p2ph.SetDeviceAttribute ("DataRate", StringValue ("10Mbps"));
  p2ph.SetChannelAttribute ("Delay", StringValue ("5ms"));
  NetDeviceContainer devs = p2ph.Install (n);
\end{lstlisting}

\section{LTE Module}
\textbf{LteHelper}
\label{sec:ltehelpercode}

\begin{lstlisting}[language=myC++,caption={LteHelper usage}, captionpos=b]
  // Create LteHelper and the nodes
  Ptr<LteHelper> lteHelper = CreateObject<LteHelper> ();
  NodeContainer enbNodes;
  enbNodes.Create (1);
  NodeContainer ueNodes;
  ueNodes.Create (2);

  // Set the mobility model
  MobilityHelper mobility;
  mobility.SetMobilityModel ("ns3::ConstantPositionMobilityModel");
  mobility.Install (enbNodes);
  mobility.SetMobilityModel ("ns3::ConstantPositionMobilityModel");
  mobility.Install (ueNodes);

  // Install NetDevices to the nodes
  NetDeviceContainer enbDevs;
  enbDevs = lteHelper->InstallEnbDevice (enbNodes);
  NetDeviceContainer ueDevs;
  ueDevs = lteHelper->InstallUeDevice (ueNodes);  

  // Attach UEs to the eNodeB
  lteHelper->Attach (ueDevs, enbDevs.Get (0));
\end{lstlisting}

\textbf{Network Attachment}

\begin{lstlisting}[language=myC++,caption={UE Automatic Attachment}, captionpos=b]
  lteHelper->Attach (ueDevs); // attach one or more UEs to a strongest cell
  lteHelper->Attach (ueDevs, enbDev); // attach one or more UEs to a single eNodeB
\end{lstlisting}


\textbf{EpcHelper}

\begin{lstlisting}[language=myC++,caption={Enable Evolved Packet Core}, captionpos=b]
  Ptr<LteHelper> lteHelper = CreateObject<LteHelper> ();
  Ptr<PointToPointEpcHelper> epcHelper = CreateObject<PointToPointEpcHelper> ();
  lteHelper->SetEpcHelper (epcHelper);

  // To access PGW or SGW 
  Ptr<Node> pgw = epcHelper->GetPgwNode ();
  Ptr<Node> sgw = epcHelper->GetSgwNode ();
\end{lstlisting}

\textbf{MAC}

\begin{lstlisting}[language=myC++,caption={MAC Scheduler}, captionpos=b]
  Ptr<LteHelper> lteHelper = CreateObject<LteHelper> ();
  lteHelper->SetSchedulerType ("ns3::FdMtFfMacScheduler");    // FD-MT scheduler
  lteHelper->SetSchedulerType ("ns3::TdMtFfMacScheduler");    // TD-MT scheduler
  lteHelper->SetSchedulerType ("ns3::TtaFfMacScheduler");     // TTA scheduler
  lteHelper->SetSchedulerType ("ns3::FdBetFfMacScheduler");   // FD-BET scheduler
  lteHelper->SetSchedulerType ("ns3::TdBetFfMacScheduler");   // TD-BET scheduler
  lteHelper->SetSchedulerType ("ns3::FdTbfqFfMacScheduler");  // FD-TBFQ scheduler
  lteHelper->SetSchedulerType ("ns3::TdTbfqFfMacScheduler");  // TD-TBFQ scheduler
  lteHelper->SetSchedulerType ("ns3::PssFfMacScheduler");     //PSS schedulerUIntegerValue(yourvalue)); 
\end{lstlisting}

\textbf{AMC \& CQI}

\begin{lstlisting}[language=myC++,caption={AMC Model}, captionpos=b]
  // Piro
  Config::SetDefault ("ns3::LteAmc::AmcModel", EnumValue (LteAmc::PiroEW2010));
  // Physical error model
  Config::SetDefault ("ns3::LteAmc::AmcModel", EnumValue (LteAmc::MiErrorModel));
\end{lstlisting}

\textbf{Building}


\begin{lstlisting}[language=myC++,caption={Mobility Model}, captionpos=b]
  MobilityHelper mobility;
  mobility.SetMobilityModel ("ns3::ConstantPositionMobilityModel");

  Ptr<Building> b = CreateObject <Building> ();
  // Box (xmin, xmax, ymin, ymax, zmin, zmax)
  b->SetBoundaries (Box (0.0, 10.0, 0.0, 20.0, 0.0, 20.0));
  b->SetBuildingType (Building::Residential);
  b->SetExtWallsType (Building::ConcreteWithWindows);
  b->SetNFloors (3);
  b->SetNRoomsX (3);
  b->SetNRoomsY (2);

  mobility.Install (ueNodes);
  mobility.Install (enbNodes);
  BuildingsHelper::Install (ueNodes);
  BuildingsHelper::Install (enbNodes);
\end{lstlisting}

\begin{lstlisting}[language=myC++,caption={Pathloss Model}, captionpos=b]
  Ptr<HybridBuildingsPropagationLossModel> propagationLossModel = CreateObject<HybridBuildingsPropagationLossModel> ();
  lteHelper->SetAttribute ("PathlossModel", StringValue ("ns3::HybridBuildingsPropagationLossModel"));
  lteHelper->SetPathlossModelAttribute ("ShadowSigmaExtWalls",   DoubleValue (1));
  lteHelper->SetPathlossModelAttribute ("ShadowSigmaOutdoor",    DoubleValue (0));
  lteHelper->SetPathlossModelAttribute ("ShadowSigmaIndoor",     DoubleValue (0));
\end{lstlisting}

\textbf{AntennaModel \& MIMO}

\begin{lstlisting}[language=myC++,caption={Antenna \& MIMO Model}, captionpos=b]
  lteHelper->SetEnbAntennaModelType ("ns3::CosineAntennaModel");
  lteHelper->SetEnbAntennaModelAttribute ("Orientation",          DoubleValue (0.0));
  lteHelper->SetEnbAntennaModelAttribute ("Beamwidth",            DoubleValue (70));
  // MaxGain in dBs
  lteHelper->SetEnbAntennaModelAttribute ("MaxGain",              DoubleValue (0.0));
  // Set the MIMO transmission mode
  Config::SetDefault ("ns3::LteEnbRrc::DefaultTransmissionMode", UintegerValue (2)); // MIMO Spatial Multiplexity (2 layers)
\end{lstlisting}

\textbf{Radio Environment Maps}

\begin{lstlisting}[language=myC++,caption={Radio Environment Maps helper}, captionpos=b]
  Ptr<RadioEnvironmentMapHelper> remHelper = CreateObject<RadioEnvironmentMapHelper> ();
  remHelper->SetAttribute ("Channel", PointerValue (lteHelper->GetDownlinkSpectrumChannel ()));
  remHelper->SetAttribute ("OutputFile", StringValue ("rem.out"));
  remHelper->SetAttribute ("XMin", DoubleValue (-400.0));
  remHelper->SetAttribute ("XMax", DoubleValue (400.0));
  remHelper->SetAttribute ("XRes", UintegerValue (100));
  remHelper->SetAttribute ("YMin", DoubleValue (-300.0));
  remHelper->SetAttribute ("YMax", DoubleValue (300.0));
  remHelper->SetAttribute ("YRes", UintegerValue (75));
  remHelper->SetAttribute ("Z", DoubleValue (0.0));
  remHelper->SetAttribute ("UseDataChannel", BooleanValue (true));
  remHelper->SetAttribute ("RbId", IntegerValue (10));
  remHelper->Install ();
\end{lstlisting}

\textbf{Parameter Configuration}

At the beginning of the \texttt{main} function, include:

\begin{lstlisting}[language=myC++, caption={Configuration parameters}, captionpos=b]
  Config::SetDefault ("ns3::ConfigStore::Filename", StringValue ("sim-input.txt"));
  Config::SetDefault ("ns3::ConfigStore::Mode", StringValue ("Load"));
  CommandLine cmd (__FILE__);
  cmd.Parse (argc, argv);
  ConfigStore inputConfig;
  inputConfig.ConfigureDefaults ();
  cmd.Parse (argc, argv);
\end{lstlisting}

Example input file

\begin{lstlisting}[language=myshell, caption={Configuration parameters}, captionpos=b]
  default ns3::LteHelper::Scheduler "ns3::PfFfMacScheduler"
  default ns3::LteHelper::PathlossModel "ns3::FriisSpectrumPropagationLossModel"
  default ns3::LteEnbNetDevice::DlBandwidth "25"
  default ns3::LteEnbNetDevice::DlEarfcn "100"
  default ns3::LteEnbNetDevice::UlEarfcn "18100"
  default ns3::LteUePhy::TxPower "10"
  default ns3::LteEnbPhy::TxPower "30"
  default ns3::LteEnbRrc::SrsPeriodicity "40"
  default ns3::TcpSocket::SndBufSize "524280"
  default ns3::TcpSocket::RcvBufSize "524280"
  global RngSeed "24"
  global simTime "10.0"
  global nRB "100"
\end{lstlisting}

\clearpage

\section{DASHjs}
\label{sec:dashjs}
\begin{lstlisting}[language=myC++, caption={DASHjs.h}, captionpos=b]
  #ifndef DASH_JS_H
  #define DASH_JS_H
  
  #include "abr-algorithm.h"
  #include "abr-variables.h"
  #include "abr-client.h"
  
  namespace ns3 {
    
  // 0   one bitrate
  // 1   set placeholder buffer such that we download fragments at most recently measured throughput.
  // 2   buffer is ready for using BOLA
  constexpr uint16_t BOLA_STATE_ONE_BITRATE = 0;
  constexpr uint16_t BOLA_STATE_STARTUP = 1;
  constexpr uint16_t BOLA_STATE_STEADY = 2;
  
  constexpr int32_t NO_CHANGE = -1;

  namespace PRIORITY {
    // The priority can have these values
    // 0.5  default priority
    // 1    strong priority
    // 0    weak priority
    constexpr double DEFAULT = 0.5;
    constexpr double STRONG = 1;
    constexpr double WEAK = 0;
  }
  
  struct BolaState {
    uint16_t              state;
    uint32_t              stableBufferTime;
    uint32_t              lastQuality;
    double                Vp;
    double                gp;
    std::vector<double>   utilities;
    std::vector<double>   bitrates;
  };
  
  struct SwitchRequest {
    double    priority;
    int32_t   quality;
  };
  
  class DASHjs : public AbrAlgorithm
  {
  public:
    DASHjs ();
    DASHjs (uint32_t bufferSize);
    /**
    * \return the next quality
    */
    uint16_t GetNextQlty ();
    /**
    * \brief    check de DASH.js rules, similar to ABRRulesCollection.js
    * \return   a list of tasks for the client to schedule
    */
    std::vector<AbrTask>  CheckRules (uint16_t     currentQlty,
                                      uint32_t     segmentDuration,
                                      uint32_t     segIndex,
                                      double       currentBw,
                                      Time         dlStartTS,
                                      PlayerStates state,
                                      std::vector<SegmentInfo> buffer);
  
    Representation GetNextRep ();
  
    // Auxiliary Functions
    SwitchRequest GetMinSwitchRequest (std::vector<SwitchRequest> requests);
    SwitchRequest CreateSwitchRequest (double priority, int32_t quality);
    SwitchRequest CreateSwitchRequest (int32_t quality);
  
  private:
    // Rule
    SwitchRequest ThroughputRule ();
  
    void      UpdateAverageEwma ();
    double    GetSafeAverageThroughput ();
    uint32_t  GetQualityForBitrate (double bitrate);
  
    // BOLA
    SwitchRequest BolaRule ();
    uint32_t      MinBufferLevelForQuality (uint32_t quality);
    uint32_t      GetQualityFromBufferLevel ();
    void          GetBolaState ();
    void          GetInitialBolaState ();
    double        GetAverageThroughput ();
    std::vector<double>   CalculateBolaParameters (uint32_t stableBufferTime, std::vector<double> bitrates, std::vector<double> utilities);
    std::vector<double>   UtilityFromBitrates (std::vector<double> bitrates);
    std::vector<double>   NormalizeUtility (std::vector<double> utilities);
  
    // Variables
    uint16_t m_currentQlty; // current buffer Quality
    uint16_t m_nextQlty; // next quality to request
    uint32_t m_segmentDuration; // segment duration in ms
    uint32_t m_segIndex; // index of the buffer playing
    uint32_t m_bufferSize; // maximum buffer size
    double   m_timeWatched; // in milliseconds
    double   m_currentBw;  // current bandwidth
    double   m_averageBw; // estimation of average bandwidth
    double   m_slowEWMA; // slow Exponentially Weighted Moving Average
    double   m_fastEWMA; // fast Exponentially Weighted Moving Average
    double   m_slowAlpha; // alpha factor for slow EWMA
    double   m_fastAlpha; // alpha factor for fast EWMA
    double   m_bandwidthSafetyFactor; // safety factor
    Time     m_dlStartTS; // time stamp of one segment starting to download
    bool     m_firstSegment; // if it is the first segment
    PlayerStates m_state; // actual state of the player
    std::vector <SegmentInfo> m_buffer; // buffer of the segments downloaded
    std::vector <Representation> m_representations;
  
    // BOLA variables
    BolaState m_bolaState;
    uint32_t  m_placeHolderBuffer;
    double    m_delay;
  };
  
  } // namespace ns3
  
  #endif /* DASH_ALGO_H */
\end{lstlisting}

\begin{lstlisting}[language=myC++, caption={DASHjs.cc}, captionpos=b]
  #include "DASHjs.h"
  #include <math.h>
  #include <cmath>
  #include <limits>
  #include <algorithm>
  
  namespace ns3 {
  NS_LOG_COMPONENT_DEFINE ("DASHjs");
  
  NS_OBJECT_ENSURE_REGISTERED (DASHjs);
  
  DASHjs::DASHjs (uint32_t bufferSize)
  {
    m_bufferSize = bufferSize;
    m_averageBw = 0.0;
    m_slowEWMA = 0.0;
    m_fastEWMA = 0.0;
    m_slowAlpha = 0.1;
    m_fastAlpha = 0.5;
    m_bandwidthSafetyFactor = 0.7;
    m_firstSegment = true;
    m_bolaState.state = -1;
    m_placeHolderBuffer = 0;
    m_delay = 0.0;
    m_representations = AbrVariables::GetRepresentations ();
  }
  
  uint16_t
  DASHjs::GetNextQlty ()
  {
    return m_nextQlty;
  }
  
  std::vector<AbrTask>
  DASHjs::CheckRules (uint16_t     currentQlty,
                      uint32_t     segmentDuration,
                      uint32_t     segIndex,
                      double       currentBw,
                      Time         dlStartTS,
                      PlayerStates state,
                      std::vector<SegmentInfo> buffer)
  {
    NS_LOG_FUNCTION (this);
    m_currentQlty = currentQlty;
    m_segmentDuration = segmentDuration;
    m_segIndex = segIndex;
    m_dlStartTS = dlStartTS;
    m_currentBw = currentBw;
    m_buffer = buffer;
    m_state = state;
  
    std::vector<SwitchRequest> requests;
    std::vector<AbrTask> tasks;
    requests.push_back (ThroughputRule ());
    requests.push_back (BolaRule ());
  
    SwitchRequest request = GetMinSwitchRequest (requests);
    if (request.quality != NO_CHANGE) {
      m_nextQlty = request.quality;
  
      NS_LOG_INFO ("Change Quality to " << AbrVariables::GetQuality (m_nextQlty));
  
      tasks.push_back (AbrVariables::CreateTask (Seconds (m_delay), AbrTask::GetNextRep));
      tasks.push_back (AbrVariables::CreateTask (Seconds (m_delay + 0.0001), AbrTask::SendRequest));
  
      m_delay = 0.0;
    }
    return tasks;
  }
  
  SwitchRequest
  DASHjs::ThroughputRule ()
  {
    SwitchRequest switchRequest = CreateSwitchRequest (NO_CHANGE);
    if (m_firstSegment && m_segIndex == 0)
    {
      NS_LOG_INFO ("First Segment");
      m_firstSegment = false;
      switchRequest.quality = 0;
      return switchRequest;
    }
    UpdateAverageEwma ();
    double average = GetSafeAverageThroughput ();
    switchRequest.quality = GetQualityForBitrate (average);
    return switchRequest;
  }
  
  SwitchRequest
  DASHjs::BolaRule ()
  {
    SwitchRequest switchRequest = CreateSwitchRequest (NO_CHANGE);
    GetBolaState ();
    if (m_bolaState.state == BOLA_STATE_ONE_BITRATE) {
      NS_LOG_INFO ("BOLA_STATE_ONE_BITRATE");
      switchRequest.quality = NO_CHANGE;
      return switchRequest;
    }
  
    // First segment
    if (m_firstSegment && m_segIndex == 0)
    {
      NS_LOG_INFO ("First Segment");
      m_firstSegment = false;
      switchRequest.quality = 0;
      return switchRequest;
    }
  
    uint32_t quality = 0;
    uint32_t bufferLevel = (m_buffer.size () - m_segIndex) * m_segmentDuration;
    uint32_t qualityForThroughput = 0;
    switchRequest.quality = 0;

    UpdateAverageEwma ();
    double safeThroughput = GetSafeAverageThroughput ();
  
    switch (m_bolaState.state) {
      case BOLA_STATE_STARTUP:
        NS_LOG_INFO ("BOLA_STATE_STARTUP");
        quality = GetQualityForBitrate(safeThroughput);
  
        switchRequest.quality = quality;
        m_bolaState.lastQuality = quality;

        if (bufferLevel >= 1)
        {
          m_bolaState.state = BOLA_STATE_STEADY;
        }

        break;
  
      case BOLA_STATE_STEADY:
        NS_LOG_INFO ("BOLA_STATE_STEADY");
  
        quality = GetQualityFromBufferLevel ();
  
        // BOLA-O variant
        qualityForThroughput = GetQualityForBitrate (safeThroughput);
        if (quality > m_bolaState.lastQuality && quality > qualityForThroughput)
        {
          // to avoid oscillations
          quality = std::max (qualityForThroughput, m_bolaState.lastQuality);
        }

        switchRequest.quality = quality;
        m_bolaState.lastQuality = quality;

        break;
  
      default:
        NS_LOG_INFO ("BOLA ABR Rule Bad State");
        quality = GetQualityForBitrate(safeThroughput);
        m_bolaState.state = BOLA_STATE_STARTUP;
  
        break;
    }
    return switchRequest;
  }
  
  void
  DASHjs::GetBolaState ()
  {
    if (m_bolaState.state > 2)
    {
      GetInitialBolaState ();
    }
  }
  
  void
  DASHjs::GetInitialBolaState ()
  {
    NS_LOG_INFO ("Initial BOLA state");
    std::vector<double> bitrates;
    bitrates = AbrVariables::GetBitratesInKbps ();
  
    std::vector<double> utilities = UtilityFromBitrates (bitrates);
    std::vector<double> normalizedUtilities = NormalizeUtility (utilities);
    uint32_t stableBufferTime = 12; // DEFAULT\_MIN\_BUFFER\_TIME;
    // uint32\_t stableBufferTime = 20; // DEFAULT\_MIN\_BUFFER\_TIME\_FAST\_SWITCH;
    std::vector<double> params = CalculateBolaParameters (stableBufferTime, bitrates, normalizedUtilities);
  
    if (params.size () <= 0)
    {
      m_bolaState.state = BOLA_STATE_ONE_BITRATE;
    }
    else
    {
      m_bolaState.state = BOLA_STATE_STARTUP;
  
      m_bolaState.bitrates = bitrates;
      m_bolaState.utilities = normalizedUtilities;
      m_bolaState.stableBufferTime = stableBufferTime;
      m_bolaState.Vp = params[0];
      m_bolaState.gp = params[1];
  
      m_bolaState.lastQuality = 0;
    }
  }
  
  std::vector<double>
  DASHjs::NormalizeUtility (std::vector<double> utilities)
  {
    std::vector<double> normalized;;
    double offset = -utilities[0];
    for (std::vector<double>::iterator it = utilities.begin();
         it != utilities.end(); ++it) {
  
      double n = *it + offset;
      normalized.push_back (n);
      NS_LOG_INFO (n);
    }
    return normalized;
  }
  
  std::vector<double>
  DASHjs::UtilityFromBitrates (std::vector<double> bitrates)
  {
    std::vector<double> utilities;
    for (std::vector<double>::iterator it = bitrates.begin();
         it != bitrates.end(); ++it) {
      double u = log(*it);
      utilities.push_back (u);
      NS_LOG_INFO (this << " " << log(*it));
    }
    return utilities;
  }
  
  std::vector<double>
  DASHjs::CalculateBolaParameters (uint32_t stableBufferTime, std::vector<double> bitrates, std::vector<double> utilities)
  {
    std::vector<double> params;
  
    const uint32_t MINIMUM_BUFFER_S = 10000;
    const uint32_t MINIMUM_BUFFER_PER_BITRATE_LEVEL_S = 2000;
    uint32_t nBitrates = bitrates.size ();
    uint32_t bufferTime = std::max (stableBufferTime,
      MINIMUM_BUFFER_S + MINIMUM_BUFFER_PER_BITRATE_LEVEL_S * nBitrates);
  
    double gp = (utilities.back () - 1) / (bufferTime / MINIMUM_BUFFER_S - 1);
    double Vp = MINIMUM_BUFFER_S / gp;
    NS_LOG_INFO ("gp: " <<gp << " u: " << utilities.back() <<" buf: "<< bufferTime);
  
    params.insert (params.begin (), Vp);
    params.insert (params.begin () + 1, gp);
    return params;
  }
  
  uint32_t
  DASHjs::MinBufferLevelForQuality (uint32_t quality)
  {
    uint32_t qBitrate = m_bolaState.bitrates[quality];
    uint32_t qUtility = m_bolaState.utilities[quality];
  
    uint32_t min = 0;
  
    for (uint16_t i = quality - 1; i > 0; --i)
    {
      NS_LOG_INFO (i);
      if (m_bolaState.utilities[i] < m_bolaState.utilities[quality])
      {
        uint32_t iBitrate = m_bolaState.bitrates[i];
        uint32_t iUtility = m_bolaState.utilities[i];
  
        uint32_t level = m_bolaState.Vp * (m_bolaState.gp + (qBitrate * iUtility - iBitrate * qUtility) / (qBitrate - iBitrate));
        min = std::max (min, level);
      }
    }
    return min;
  }
  
  // Main function
  uint32_t
  DASHjs::GetQualityFromBufferLevel ()
  {
    uint32_t bitrateCount = m_bolaState.bitrates.size ();
    uint32_t quality = 0;
    uint32_t bufferLevel = (m_buffer.size () - m_segIndex) * m_segmentDuration;
    double score = NAN;
    for (uint16_t i = 0; i < bitrateCount; ++i)
    {
      double s = (m_bolaState.Vp * (m_bolaState.utilities[i] + m_bolaState.gp)
      - bufferLevel) / (m_bolaState.bitrates[i]);
      if (std::isnan(score) || s > score)
      {
        NS_LOG_INFO ("s: " << s << " Vp: "<< m_bolaState.Vp << " u: "
        << m_bolaState.utilities[i]
        << " gp: " << m_bolaState.gp << " level: " << bufferLevel
        << " bitrate: " << m_bolaState.bitrates[i]);
        score = s;
        quality = i;
      }
    }
    NS_LOG_INFO (quality << " " << AbrVariables::GetQuality(quality) << " "
    << m_representations[quality].bitrate << "Buffer Level: " << bufferLevel);
    return quality;
  }
  
  double
  DASHjs::GetAverageThroughput ()
  {
    return m_averageBw;
  }

  SwitchRequest
  DASHjs::GetMinSwitchRequest (std::vector<SwitchRequest> requests)
  {
    SwitchRequest newSwitchReq = CreateSwitchRequest (NO_CHANGE);
    int32_t newQuality = -1;
    std::map<double, int32_t> values;
  
    if (requests.size() == 0)
    {
      return newSwitchReq;
    }
    else if (requests.size() == 1)
    {
      return requests.back ();
    }
  
    values.insert (std::pair<double, int32_t>(PRIORITY::STRONG, NO_CHANGE));
    values.insert (std::pair<double, int32_t>(PRIORITY::WEAK, NO_CHANGE));
    values.insert (std::pair<double, int32_t>(PRIORITY::DEFAULT, NO_CHANGE));
  
    for (std::vector<SwitchRequest>::iterator it = requests.begin ();
         it != requests.end (); ++it) {
      SwitchRequest req = *it;
      if (req.quality != NO_CHANGE) {
        if (values.at (req.priority) == NO_CHANGE ||
            values.at (req.priority) > req.quality) {
          NS_LOG_INFO (req.quality);
          values.at (req.priority) = req.quality;
        }
      }
    }
  
    if (values.at (PRIORITY::WEAK) != NO_CHANGE) {
      newQuality = values.at (PRIORITY::WEAK);
    }
  
    if (values.at (PRIORITY::DEFAULT) != NO_CHANGE) {
      newQuality = values.at (PRIORITY::DEFAULT);
    }
  
    if (values.at (PRIORITY::STRONG) != NO_CHANGE) {
      newQuality = values.at (PRIORITY::STRONG);
    }
  
    if (newQuality > -1) {
      newSwitchReq = CreateSwitchRequest (newQuality);
      NS_LOG_INFO (newQuality);
      NS_LOG_INFO ("SwitchRequest to quality" << AbrVariables::GetQuality (m_nextQlty));
    }
  
    return newSwitchReq;
  }
  
  SwitchRequest
  DASHjs::CreateSwitchRequest (double priority, int32_t quality) {
    SwitchRequest req;
    if (priority != 0 || priority != 0.5 || priority != 1) {
       // priority by default
       std::cout << priority << std::endl;
       req.priority = PRIORITY::DEFAULT;
     } else {
       req.priority = priority;
     }
    req.priority = priority;
    req.quality = quality;
    NS\_LOG\_INFO (req.priority << " " << req.quality);
    return req;
  }
  
  SwitchRequest
  DASHjs::CreateSwitchRequest (int32_t quality) {
    SwitchRequest req;
    req.priority = PRIORITY::DEFAULT;
    req.quality = quality;
    NS_LOG_INFO (req.priority << " " << req.quality);
    return req;
  }
  
  uint32_t
  DASHjs::GetQualityForBitrate (double bitrate)
  {
    Representation rep;
    if (m_representations.size () < 2) return 0;
  
    for (uint16_t j=m_representations.size () - 1; j>0; j--)
    {
      rep = m_representations[j];
      // bitrates are in bps
      if (bitrate > rep.bitrate) {
        NS_LOG_INFO (j << " " << AbrVariables::GetQuality(j) << " " << m_representations[j].bitrate << " " << bitrate);
        return j;
      }
    }
    return 0;
  }
  
  void
  DASHjs::UpdateAverageEwma ()
  {
    double newSample = 0;
    if (m_buffer.size () != 0)
    {
      newSample = m_buffer.back ().dlBandwidth;
    }
  
    if (m_fastEWMA == 0 || m_slowEWMA == 0)
    {
      m_fastEWMA = newSample;
      m_slowEWMA = newSample;
    }
    else
    {
      m_fastEWMA = newSample * m_fastAlpha + m_fastEWMA * (1 - m_fastAlpha);
      m_slowEWMA = newSample * m_slowAlpha + m_slowEWMA * (1 - m_slowAlpha);
    }
    m_averageBw = std::min (m_slowEWMA, m_fastEWMA);
  }
  
  double
  DASHjs::GetSafeAverageThroughput ()
  {
    double average = m_averageBw;
    if (average != 0) {
      average *= m_bandwidthSafetyFactor;
    }
    return average;
  }
  
  Representation
  DASHjs::GetNextRep ()
  {
    Representation rep = AbrVariables::GetRep (m_currentQlty);
    return rep;
  }
  
  } // namespace ns3
  
\end{lstlisting}
