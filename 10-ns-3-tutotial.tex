\chapter{ns-3 Tutorial} 
\label{chap:ns3tuto}

\section{Getting Started}

The prerequisites for the \textit{ns-3} release version 3.32 are the following tools:

\begin{table}[h]
  \centering
  \begin{tabular}{@{}ll@{}}
  \toprule
  Prerequisite & Package/version                             \\ \midrule 
  C++ compiler & \texttt{clang++} or \texttt{g++} (\texttt{g++} version 4.9 or greater) \\
  Python       & \texttt{python3}  version \textgreater{}= 3.5         \\
  Git          & any recent version                          \\ 
  tar          & any recent version                          \\ 
  bunzip2      & any recent version                          \\ \bottomrule \\
  \end{tabular}
  \caption{Prerequisites for ns-3}
\end{table}

Start by downloading the source archive from \href{https://www.nsnam.org/release/ns-allinone-3.32.tar.bz2}{nsnam}
or \href{https://gitlab.com/nsnam/ns-3-allinone.git}{gitlab}. Then build \textit{ns-3} with \texttt{build.py}:

\begin{lstlisting}[language=myshell,caption={Download and installation of ns-3}, captionpos=b]
  # Download from nsnam
  $ cd
  $ mkdir workspace
  $ cd workspace
  $ wget https://www.nsnam.org/release/ns-allinone-3.32.tar.bz2
  $ |\color{myblue}tar| xjf ns-allinone-3.32.tar.bz2
  $ cd ns-allinone-3.32
  # Building ns-3
  $ ./build.py --enable-examples --enable-tests
  # Running a script
  # Create or copy a script to the scratch directory
  $ cp examples/tutorial/first.cc scratch/myfirst.cc
  $ ./waf --run scratch/myfirst
\end{lstlisting}

\textbf{Logging Module}

Enable logging:

\begin{lstlisting}[escapechar=@, language=myshell,caption={Enabling logging in ns-3}, captionpos=b]
  $ export 'NS_LOG=UdpEchoClientApplication=level_all|prefix_func|prefix_time:UdpEchoServerApplication=level_all|prefix_func|prefix_time'
\end{lstlisting}

To disable logging simply type:

\begin{lstlisting}[escapechar=@, language=myshell,caption={Disabling logging in ns-3}, captionpos=b]
  $ export NS_LOG=
\end{lstlisting}

\textbf{Command Line Arguments}

An example of a command could be like this:

\begin{lstlisting}[escapechar=@, language=myshell,caption={Command line arguments}, captionpos=b]
  # To check help 
  $ ./waf --run "scratch/myfirst --PrintHelp"
  # To check variables for PointToPointNetDevice
  $ ./waf --run "scratch/myfirst --PrintAttributes=ns3::PointToPointNetDevice"
  # We set the Datarate to 5Mbps
  $ ./waf --run "scratch/myfirst --ns3::PointToPointNetDevice::DataRate=5Mbps"
\end{lstlisting}

\textbf{ASCII Tracing}

To enable ASCII Tracing, right before the call to \texttt{Simulator::Run ()}, add the following lines
of code:

\begin{lstlisting}[escapechar=@, language=myC++,caption={ASCII tracing}, captionpos=b]
  AsciiTraceHelper ascii;
  pointToPoint.EnableAsciiAll (ascii.CreateFileStream ("out.tr"));
\end{lstlisting}

This will generate the output from \texttt{pointToPoint} to a file named \texttt{out.tr}.

\textbf{PCAP Tracing}

To enable \textit{pcap} tracing simply add:
\begin{lstlisting}[escapechar=@, language=myC++,caption={PCAP tracing}, captionpos=b]
  pointToPoint.EnablePcapAll ("myfirst");
\end{lstlisting}

\textbf{Sockets}

Creating a socket:

\begin{lstlisting}[language=myC++, caption={ns-3 Socket programming}, captionpos=b]
  Ptr<Node> node;
  // Create a TCP socket
  Ptr<Socket> mySocket = Socket::CreateSocket (node, TcpSocketFactory::GetTypeId ());
  // Functions
  mySocket->Bind();
  mySocket->Connect( ... );
  mySocket->Send( ... );
  mySocket->Recv( ... );
  mySocket->Close();
\end{lstlisting}

For callbacks:

\begin{lstlisting}[language=myC++, caption={Socket callbacks}, captionpos=b]
  mySocket->SetConnectCallback (
    MakeCallback (&MyClass::ConnectionSucceeded, this),
    MakeCallback (&MyClass::ConnectionFailed, this)
  );
  mySocket->SetSendCallback (MakeCallback (
    &MyClass::HandleSend, this));
  mySocket->SetRecvCallback (MakeCallback (
    &MyClass::HandleRead, this));
\end{lstlisting}

\textbf{PointToPoint NetDevice}

\begin{lstlisting}[language=myC++, caption={PointToPointHelper}, captionpos=b]
  NodeContainer n;
  n.Create (2);
  PointToPointHelper p2ph;
  p2ph.SetDeviceAttribute ("DataRate", StringValue ("10Mbps"));
  p2ph.SetChannelAttribute ("Delay", StringValue ("5ms"));
  NetDeviceContainer devs = p2ph.Install (n);
\end{lstlisting}

\section{LTE Module}
\textbf{LteHelper}
\label{sec:ltehelpercode}

\begin{lstlisting}[language=myC++,caption={LteHelper usage}, captionpos=b]
  // Create LteHelper and the nodes
  Ptr<LteHelper> lteHelper = CreateObject<LteHelper> ();
  NodeContainer enbNodes;
  enbNodes.Create (1);
  NodeContainer ueNodes;
  ueNodes.Create (2);

  // Set the mobility model
  MobilityHelper mobility;
  mobility.SetMobilityModel ("ns3::ConstantPositionMobilityModel");
  mobility.Install (enbNodes);
  mobility.SetMobilityModel ("ns3::ConstantPositionMobilityModel");
  mobility.Install (ueNodes);

  // Install NetDevices to the nodes
  NetDeviceContainer enbDevs;
  enbDevs = lteHelper->InstallEnbDevice (enbNodes);
  NetDeviceContainer ueDevs;
  ueDevs = lteHelper->InstallUeDevice (ueNodes);  

  // Attach UEs to the eNodeB
  lteHelper->Attach (ueDevs, enbDevs.Get (0));
\end{lstlisting}

\textbf{Network Attachment}

\begin{lstlisting}[language=myC++,caption={UE Automatic Attachment}, captionpos=b]
  lteHelper->Attach (ueDevs); // attach one or more UEs to a strongest cell
  lteHelper->Attach (ueDevs, enbDev); // attach one or more UEs to a single eNodeB
\end{lstlisting}


\textbf{EpcHelper}

\begin{lstlisting}[language=myC++,caption={Enable Evolved Packet Core}, captionpos=b]
  Ptr<LteHelper> lteHelper = CreateObject<LteHelper> ();
  Ptr<PointToPointEpcHelper> epcHelper = CreateObject<PointToPointEpcHelper> ();
  lteHelper->SetEpcHelper (epcHelper);

  // To access PGW or SGW 
  Ptr<Node> pgw = epcHelper->GetPgwNode ();
  Ptr<Node> sgw = epcHelper->GetSgwNode ();
\end{lstlisting}

\textbf{MAC}

\begin{lstlisting}[language=myC++,caption={MAC Scheduler}, captionpos=b]
  Ptr<LteHelper> lteHelper = CreateObject<LteHelper> ();
  lteHelper->SetSchedulerType ("ns3::FdMtFfMacScheduler");    // FD-MT scheduler
  lteHelper->SetSchedulerType ("ns3::TdMtFfMacScheduler");    // TD-MT scheduler
  lteHelper->SetSchedulerType ("ns3::TtaFfMacScheduler");     // TTA scheduler
  lteHelper->SetSchedulerType ("ns3::FdBetFfMacScheduler");   // FD-BET scheduler
  lteHelper->SetSchedulerType ("ns3::TdBetFfMacScheduler");   // TD-BET scheduler
  lteHelper->SetSchedulerType ("ns3::FdTbfqFfMacScheduler");  // FD-TBFQ scheduler
  lteHelper->SetSchedulerType ("ns3::TdTbfqFfMacScheduler");  // TD-TBFQ scheduler
  lteHelper->SetSchedulerType ("ns3::PssFfMacScheduler");     //PSS schedulerUIntegerValue(yourvalue)); 
\end{lstlisting}

\textbf{AMC \& CQI}

\begin{lstlisting}[language=myC++,caption={AMC Model}, captionpos=b]
  // Piro
  Config::SetDefault ("ns3::LteAmc::AmcModel", EnumValue (LteAmc::PiroEW2010));
  // Physical error model
  Config::SetDefault ("ns3::LteAmc::AmcModel", EnumValue (LteAmc::MiErrorModel));
\end{lstlisting}

\textbf{Building}


\begin{lstlisting}[language=myC++,caption={Mobility Model}, captionpos=b]
  MobilityHelper mobility;
  mobility.SetMobilityModel ("ns3::ConstantPositionMobilityModel");

  Ptr<Building> b = CreateObject <Building> ();
  // Box (xmin, xmax, ymin, ymax, zmin, zmax)
  b->SetBoundaries (Box (0.0, 10.0, 0.0, 20.0, 0.0, 20.0));
  b->SetBuildingType (Building::Residential);
  b->SetExtWallsType (Building::ConcreteWithWindows);
  b->SetNFloors (3);
  b->SetNRoomsX (3);
  b->SetNRoomsY (2);

  mobility.Install (ueNodes);
  mobility.Install (enbNodes);
  BuildingsHelper::Install (ueNodes);
  BuildingsHelper::Install (enbNodes);
\end{lstlisting}

\begin{lstlisting}[language=myC++,caption={Pathloss Model}, captionpos=b]
  Ptr<HybridBuildingsPropagationLossModel> propagationLossModel = CreateObject<HybridBuildingsPropagationLossModel> ();
  lteHelper->SetAttribute ("PathlossModel", StringValue ("ns3::HybridBuildingsPropagationLossModel"));
  lteHelper->SetPathlossModelAttribute ("ShadowSigmaExtWalls",   DoubleValue (1));
  lteHelper->SetPathlossModelAttribute ("ShadowSigmaOutdoor",    DoubleValue (0));
  lteHelper->SetPathlossModelAttribute ("ShadowSigmaIndoor",     DoubleValue (0));
\end{lstlisting}

\textbf{AntennaModel \& MIMO}

\begin{lstlisting}[language=myC++,caption={Antenna \& MIMO Model}, captionpos=b]
  lteHelper->SetEnbAntennaModelType ("ns3::CosineAntennaModel");
  lteHelper->SetEnbAntennaModelAttribute ("Orientation",          DoubleValue (0.0));
  lteHelper->SetEnbAntennaModelAttribute ("Beamwidth",            DoubleValue (70));
  // MaxGain in dBs
  lteHelper->SetEnbAntennaModelAttribute ("MaxGain",              DoubleValue (0.0));
  // Set the MIMO transmission mode
  Config::SetDefault ("ns3::LteEnbRrc::DefaultTransmissionMode", UintegerValue (2)); // MIMO Spatial Multiplexity (2 layers)
\end{lstlisting}

\textbf{Radio Environment Maps}

\begin{lstlisting}[language=myC++,caption={Radio Environment Maps helper}, captionpos=b]
  Ptr<RadioEnvironmentMapHelper> remHelper = CreateObject<RadioEnvironmentMapHelper> ();
  remHelper->SetAttribute ("Channel", PointerValue (lteHelper->GetDownlinkSpectrumChannel ()));
  remHelper->SetAttribute ("OutputFile", StringValue ("rem.out"));
  remHelper->SetAttribute ("XMin", DoubleValue (-400.0));
  remHelper->SetAttribute ("XMax", DoubleValue (400.0));
  remHelper->SetAttribute ("XRes", UintegerValue (100));
  remHelper->SetAttribute ("YMin", DoubleValue (-300.0));
  remHelper->SetAttribute ("YMax", DoubleValue (300.0));
  remHelper->SetAttribute ("YRes", UintegerValue (75));
  remHelper->SetAttribute ("Z", DoubleValue (0.0));
  remHelper->SetAttribute ("UseDataChannel", BooleanValue (true));
  remHelper->SetAttribute ("RbId", IntegerValue (10));
  remHelper->Install ();
\end{lstlisting}

\textbf{Parameter Configuration}

At the beginning of the \texttt{main} function, include:

\begin{lstlisting}[language=myC++, caption={Configuration parameters}, captionpos=b]
  Config::SetDefault ("ns3::ConfigStore::Filename", StringValue ("sim-input.txt"));
  Config::SetDefault ("ns3::ConfigStore::Mode", StringValue ("Load"));
  CommandLine cmd (__FILE__);
  cmd.Parse (argc, argv);
  ConfigStore inputConfig;
  inputConfig.ConfigureDefaults ();
  cmd.Parse (argc, argv);
\end{lstlisting}

Example input file

\begin{lstlisting}[language=myshell, caption={Configuration parameters}, captionpos=b]
  default ns3::LteHelper::Scheduler "ns3::PfFfMacScheduler"
  default ns3::LteHelper::PathlossModel "ns3::FriisSpectrumPropagationLossModel"
  default ns3::LteEnbNetDevice::DlBandwidth "25"
  default ns3::LteEnbNetDevice::DlEarfcn "100"
  default ns3::LteEnbNetDevice::UlEarfcn "18100"
  default ns3::LteUePhy::TxPower "10"
  default ns3::LteEnbPhy::TxPower "30"
  default ns3::LteEnbRrc::SrsPeriodicity "40"
  default ns3::TcpSocket::SndBufSize "524280"
  default ns3::TcpSocket::RcvBufSize "524280"
  global RngSeed "24"
  global simTime "10.0"
  global nRB "100"
\end{lstlisting}
