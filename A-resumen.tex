\cleardoublepage
\phantomsection
\chapter*{Resumen}
\addcontentsline{toc}{chapter}{Resumen}

El streaming de vídeo con tasa de bits adaptativa se está convirtiendo 
en la técnica más utilizada por las plataformas de vídeo en línea. 
Con la pandemia mundial \textit{COVID-19}, el streaming de vídeo se ha convertido 
en una de las principales fuentes de entretenimiento durante los confinamientos. 
De hecho, más de la mitad de la cuota de tráfico de la red se utiliza hoy en 
día para streaming de vídeo \cite{sandvine1}.

El objetivo de este Trabajo Fín de Máster (TFM) es construir un framework en \textit{ns-3},
implementado en \textit{C++}, para analizar y comparar algunas implementaciones de algoritmos de adaptación de vídeo
sobre diferentes escenarios de red. El primer paso 
es estudiar \textit{ns-3}, familiarizarse con algunos módulos de \textit{ns-3} y construir varios 
escenarios de red \textit{LTE}. El segundo paso es construir un módulo que pueda simular 
servidores y clientes de vídeo de \textit{BitRate Adaptativo (ABR)}, estudiar algunos enfoques de los algoritmos
de adaptación de la tasa de bits de vídeo e implementar dichos algoritmos, 
incluyendo soluciones basadas en el ancho de banda, en el buffer y algoritmos 
híbridos. 
Por último, podemos comparar y evaluar el rendimiento de diferentes algoritmos 
\textit{ABR} en escenarios con condiciones variables con diferentes métricas objetivas 
de \textit{QoE}.

Como resultado, se han probado diferentes implementaciones de algoritmos de adaptación con
este nuevo módulo \textit{ABR}. Aunque es evidente que tiene sus limitaciones siendo un entorno simulado.

Este proyecto se ha llevado a cabo con la cátedra Ericsson-UPM en software y sistemas.



\vfill
\textbf{Palabras clave: DASH, ABR, ns-3, streaming de video por HTTP, simulación, QoE} 


\cleardoublepage
\phantomsection
\chapter*{Abstract}
\addcontentsline{toc}{chapter}{Abstract}

Adaptive bitrate video streaming is becoming the most used technique for online
video platforms. With the \textit{COVID-19} worldwide pandemic, video streaming has become
one of the primary sources of entertainment during the shutdown. In fact, more
than half of the network traffic share today is used by video streaming \cite{sandvine1}.

The objective of this Master's Thesis is to build a framework in \textit{ns-3}, implemented
in \textit{C++}, for testing video adaptation algorithms and to compare some implementations
over different network scenarios. The first step is to study \textit{ns-3}, familiarize with
some \textit{ns-3} modules, and build various LTE network scenarios. The second step is to
build a module that can simulate \textit{ABR} video servers and clients, study some approaches
of video bitrate adaptation algorithms and implement those algorithms, including
throughput based, buffer based and hybrid solutions. Finally we can compare and 
evaluate the performance of different \textit{ABR} algorithms on scenarios with varying 
conditions with different objective \textit{QoE} metrics.

As a result, different implementations of adaptation algorithms have been tested with
this new \textit{ABR} module. Although it is evident that it has its limitations being a
simulated enviroment.

This project has been carried out with the Ericsson-UPM scholarship in software and systems.

\vfill
\textbf{Keywords: DASH, ABR, ns-3, HTTP video streaming, simulation, QoE} 