\chapter{Introduction}
\label{chap:introduction}

\section{Context}
\label{sec:context}

There is no doubt about the importance of online video streaming.
According to Sandvine \cite{sandvine1},
in 2020, 57\% of the global internet traffic was used by video streaming.
Moreover, one of the key predictions made by Cisco in 2018 \cite{cisco1}
stated that by year 2022, video traffic will make up 82\% 
of all \textit{IP} traffic.

Consequently, many challenges arise. Due to the
growth of the number and diversity of video capable connected 
devices and every time more available bandwidth and better quality 
contents, the client and the server need to adapt the video content to
the network and the devices. The technique of taking account the 
varying network conditions and computing resources of the user 
device to choose the adequate quality level is denominated as
\textit{Adaptive BitRate (ABR)}. Adaptation may be performed
monitoring different parameters such as estimated bandwidth,
client's buffer level, CPU load or screen size.


\begin{figure}[h]
  \label{fig:chart1}
    \centering
    \includegraphics[width=0.8\textwidth]{img/chart2.png}
    \caption{Global application category total traffic share during COVID-19 lockdown. Source: Sandvine \cite{sandvine1}}
  \end{figure}

The \textit{Dynamic Adaptive Streaming over HTTP (DASH)} is one of the
standards that implements adaptive bitrate video streaming and was developed
by the \textit{Moving Picture Experts Group (MPEG)} \cite{dash1}. \textit{MPEG-DASH} 
enables provisioning and delivering media using existing \textit{HTTP}-delivery 
networks supports dynamic adaptation with seamless switching. By using
\textit{HTTP}, the player will not have firewall problems, it will have better
scalability and the quality selection relays on the client and there is no
need to have session at the server.

The \textit{MPEG-DASH} standard was published in 2012 and revised in 2019 
by the \textit{International Organization for Standardization (ISO) / International 
Electrotechnical Commission (IEC)} as \textit{MPEG-DASH ISO/IEC 23009-1:2019}
\cite{ISO23009}. In addition, the \textit{3\textsuperscript{rd} Generation Partnership Project (3GPP)}
define the use of \textit{DASH} as the standard continuous delivering of multimedia
content in mobile networks, specifically in 4G such as \textit{LTE} and 5G networks.

\textit{DASH} divides the media file into small chunks or segments.
\textit{MPEG-DASH} defines the \textit{Media Presentation Description (MPD)}, 
which is an XML-structured manifest file that contains the \textit{Universal Resource 
Locators (URL)} of the segments. Different qualities are defined as representations,
the \textit{MPD} file contains information for each representation such as the
codec, bandwidth, the resolution of the video or framerate.

However, the DASH Standard \cite{ISO23009} only defines the data formats
for the media reproduction and do not provide the adaptation algorithm.
The \textit{DASH Industry Forum} \cite{dash2} provides an open source \textit{MPEG-DASH} 
player implemented in \textit{JavaScript} with different adaptation algorithms.
Similarly, \textit{hls.js} is an implementation of a \textit{HTTP Live Streaming}\footnote{HTTP
Live Streaming is a HTTP-based adaptive bitrate streaming protocol developed by Apple Inc.
 \cite{hls1}} client.

The adaptation algorithms needs to be tested in different scenarios (they can be simulated)
and tweaked to provide the maximum perceived quality by the users. Also, there are
algorithms that perform better in some specific scenarios and worse in others. The adaptation
algorithm is the responsible of avoid problems that have a negative impact
on the \textit{Quality of Experience (QoE)}. Firstly, the algorithm can overestimate
the bandwidth and it would cause a pause in the reproduction because all the 
segments in the buffer is emptied. The algorithm can also underestimate the bandwidth,
the video player requests media segments with inferior quality than the quality at which the 
bandwidth available of the network can allow. Lastly, the algorithm should avoid
constant bitrate switches result of bandwidth fluctuations, and provide a smooth and
seamless video watching experience.

The \textit{ns-3} simulator is an open-source and extensible discrete-event network simulator. 
The extensible nature of this tool allows us to develop a new module for \textit{ns-3}
mimicking the behaviour of \textit{ABR} clients and servers. With this new module, \textit{ns-3} 
will be able to simulate extreme network scenarios and test the performance of 
various adaptation algorithms.

\section{Objectives}
\label{sec:objectives}
The objectives of this thesis is to build a framework for testing \textit{ABR} adaptation
algorithms, and implement some adaptation algorithms and compare them in 
various mobile network scenarios with different objective \textit{QoE} metrics. 
In order to achieve the proposed objectives, the following steps will be proposed:

\begin{enumerate}
  \item Study and understand \textit{ns-3} and basic modules such as the core module, the
  internet module, applications module, \textit{LENA} module among others. Build basic \textit{LTE} scenarios
  tweak radio parameters, and output results.
  \item Design a new module in \text{ns-3} that simulates behaviours of \textit{ABR} clients
  and servers. Study and implement existing adaptation algorithms.
  \item Define and implement objective \textit{QoE} metrics.
  Build new \textit{LTE} scenarios and compare the performances of the implemented adaptation
  algorithms.
\end{enumerate}


\section{Structure of the thesis}
\label{sec:structure}

% The memory of this work is structured as follow:

\textbf{\textit{Chapter 1.}} Presents the context, the motivations and the objectives of this thesis.

\textbf{\textit{Chapter 2.}} The State of the Art.

\textbf{\textit{Chapter 3.}} dddd

\textbf{\textit{Chapter 4.}} dddd

\textbf{\textit{Chapter 5.}} dddd
