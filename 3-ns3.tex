\chapter{Network Simulator 3}
\label{chap:ns3}

The \textit{ns-3} simulator is an open, extensible discrete-event network simulator designed primarily 
for educational and network research purposes \cite{ns3}.

In summary, \textit{ns-3} provides models of how packet data networks work and operate, as well as a 
simulation engine that allows users to run simulation experiments. To do research that are 
more difficult or impossible to do with real systems, to examine system behavior in a highly 
controlled, reproducible setting, and to understand about how networks work.

\textit{ns-3} is a collection of modules that can be used together as well as with other software 
libraries. This tool works mainly at the command line on \textit{Linux} or \textit{MacOS} and with 
\textit{C++} and \textit{Python} programming languages and development tools.

\section{ns-3 Concepts}
\label{sec:ns3conc}
This section will go over several networking concepts that have a specific meaning in ns-3.

\begin{itemize}[topsep=0pt]
  \item[] \textbf{Node}: A \texttt{Node} in \textit{ns-3} is the basic computing device abstraction. 
  The \texttt{Node} class has methods for managing computing device representations in simulations.
  \item[] \textbf{Application}: A \textit{ns-3} application run on \textit{ns-3} \texttt{Nodes}. An 
  \texttt{Applicacion} is the basic abstraction for a user program that generates some simulated activity.  
  The \texttt{Application} class provides functions for controlling the representations of the simulated
  version of user-level applications.
  \item[] \textbf{Channel}: A \texttt{Channel} in \textit{ns-3} is an abstraction of the basic communication 
  subnetwork in which \texttt{Nodes} are connected in. It can be as simple as a wire or as complicated as a Ethernet switch.
  \item[] \textbf{Net Device}: A \texttt{NetDevice} in \textit{ns-3} simulates a \textit{Network Interface Card (NIC)}
  and the software controlling the \textit{NIC}. A \texttt{NetDevice} is installed in a \texttt{Node} to allow 
  it to communicate over \texttt{Channels} with other \texttt{Nodes} in the simulation.
  \item[] \textbf{Helpers}: Helper objects are created to make some commun tasks easier. Such as connecting
  \texttt{NetDevices} to \texttt{Nodes}, \texttt{NetDevices} to \texttt{Channels}, assigning IP addresses, etc. 
\end{itemize}

The \autoref{fig:ns3arch} shows a high level node architecture.

\begin{figure}[h]
  \centering
  \includegraphics[width=0.95\textwidth]{img/node.png}
  \caption{ns-3 High-level node architecture. Source:\cite{ns3} }
  \label{fig:ns3arch}
\end{figure}

\section{Logging Module}
Message logging is a basic feature for large softwares, and \textit{ns-3} is no different. \textit{ns-3} 
offer a complete module for message logging with configurable verbosity levels. This means that logging 
functions of specific components can be enabled and other can be disabled completely.

There are different levels of log messages of ascending verbosity defined in \textit{ns-3}:

\begin{itemize}[topsep=0pt]
  \item \textbf{\texttt{LOG\_ERROR}}: For error messages (associated function: \texttt{NS\_LOG\_ERROR}).
  \item \textbf{\texttt{LOG\_WARN}}: For warning messages (associated function: \texttt{NS\_LOG\_WARN}).
  \item \textbf{\texttt{LOG\_DEBUG}}: For relatively rare, ad-hoc debugging messages (associated function: \texttt{NS\_LOG\_DEBUG}).
  \item \textbf{\texttt{LOG\_INFO}}: For informational messages about program progress (associated function: \texttt{NS\_LOG\_INFO}).
  \item \textbf{\texttt{LOG\_FUNCTION}}: For messages describing each function called (two associated function: \texttt{NS\_LOG\_FUNCTION}
  used for member functions, and \texttt{NS\_LOG\_FUNCTION\_NOARGS}, used for static functions)).
  \item \textbf{\texttt{LOG\_LOGIC}}: For messages describing logical flow within a function (associated function: \texttt{NS\_LOG\_LOGIC}).
  \item \textbf{\texttt{LOG\_ALL}}: Log everything mentioned above (no associated function).
\end{itemize}

To enable all logs, it is as simple as modifying a shell variable. In the next example the logging for the class
\texttt{UdpEchoClientApplication} and \texttt{UdpEchoServerApplication} is enabled with all levels, the time 
and the function prefixes:

\begin{lstlisting}[escapechar=@, language=myshell,caption={Enabling logging in ns-3}, captionpos=b]
  $ export 'NS_LOG=UdpEchoClientApplication=level_all|prefix_func|prefix_time:UdpEchoServerApplication=level_all|prefix_func|prefix_time'
\end{lstlisting}

For more information with the logging modure see \cite{ns3}.

\section{Command Line Arguments}
There are local and global variables that can be changed in the command line without
editing the scripts. To be able to know what variables are available, the option \texttt{--PrintHelp} is used.


\section{Tracing System}
The main goal of the simulations is to extract and generate output, and \textit{ns-3} offers two
mechanisms for this. Also, since \textit{ns-3} is a C++ software, using \texttt{std::cout} for
output is also available.

\subsection{ASCII Tracing}
ns-3 includes helper function that encapsulates the low-level tracing system and guides you 
through the technicalities of establishing some simple packet traces. If you enable this feature, 
the output will be in ASCII files, hence the name.

To enable ASCII Tracing, right before the call to \texttt{Simulator::Run ()}, create an \texttt{AsciiTraceHelper}
and call the function \texttt{EnableAsciiAll}.
This will generate the output into the home directory of \textit{ns-3}.

\subsection{PCAP Tracing}
The \textit{ns-3} device helpers can also create \texttt{.pcap} trace files. The \textit{pcap} file
contains the packets captured during the simulation. \textit{Wireshark} or \textit{tcpdump} 
are programs capable of reading and visualizing \textit{pcap} files.

To enable \textit{pcap} tracing simply add the \texttt{EnablePcapAll} function.
And it will create various \texttt{.pcap} files in the format \texttt{"myfirst-0-0.pcap"}, meaning
the trace file for node 0 and device 0.

\section{ns-3 Modules \& Models}
In this section, modules used in this thesis will be presented, based on the official
manual from \cite{ns3}. But first, It is 
essential to understand the difference between modules and models:

% \begin{itemize}[noitemsep,topsep=0pt]
\begin{itemize}[itemsep=0pt, topsep=0pt]
  \item \textbf{Modules} are the different libraries that form \textit{ns-3}.
  \item \textbf{Models} are the simulated, abstract representations of real-life objects,
  protocols, devices, etc.
\end{itemize}

As the reader may already know, \textit{ns-3} is modular. A new module will be introduced in 
the \autoref{chap:abrmodule} as a result of this Master final project.

\subsection{Antenna Module}
The Antenna module provides a \texttt{AntennaModel} base class as an interface for radiation
pattern modelling of an antenna. Also, there are a set of classes derived from this base
class that implements types of antennas with differente radiation patterns.

\subsubsection{AntennaModel}
The \texttt{AntennaModel} uses a coordinate system as shown in the \autoref{fig:ns3antenna}. This
model uses, for a point ${p}$ in the space with Cartesian coordinates used by the \texttt{MobilityModel},
the coordinates ${(x,y,z)}$ and transforms into spherical coordinates ${(r,\theta,\phi)}$.

The radiation pattern is express as a mathematical function ${g(\theta,\phi) \rightarrow \R}$

\begin{figure}[h]
  \centering
  \includegraphics[width=0.4\textwidth]{img/antennamodel.png}
  \caption{Coordinate system of the AntennaModel. Source: nsnam \cite{ns3}}
  \label{fig:ns3antenna}
\end{figure}

\begin{itemize}[itemsep=0pt, topsep=0pt]
  \item \textbf{IsotropicAntennaModel} 
  
  The \texttt{IsotropicAntennaModel} is onmidirectional, this means that the radiation pattern have
  a 0dB gain for all direction.
  \item \textbf{CosineAntennaModel}
  
  The antenna gain of the \texttt{CosineAntennaModel} is defined as:
  \begin{equation}
    g(\phi,\theta)=cos^n \bigg( \frac{\phi-\phi_0}{2} \bigg)
  \end{equation}

  with ${\phi_0}$ as the antenna's azimuthal orientation, this is, the direction of maximum gain. And the exponential
  \begin{equation}
    n=-\frac{3}{20\,log_{10} \big( cos\frac{\phi_{3dB}}{4} \big) }
  \end{equation}

  determines the wanted 3db beamwidth ${\phi_{3dB}}$.
  \item \textbf{ParabolicAntennaModel}

  In the \texttt{ParabolicAntennaModel}, the antenna gain is determined as:
  \begin{equation}
    g(\phi,\theta)=-min \bigg( 12 \bigg( \frac{\phi-\phi_0}{\phi_{3dB}}^2 \bigg) ,A_{max} \bigg)
  \end{equation}
  where ${A_{max}}$ is the maximum attenuation in dB of the antenna. 
\end{itemize}

\subsection{Application Module}
The \texttt{Application} class is a base class for \textit{ns-3} applications. Nodes can
have one or more applications. Each node maintains a list of smart pointers (references) 
to all its applications. The \texttt{Applications} are the responsibles to create the sockets,
if needed.

There are a few implementations of \texttt{Applications} in \textit{ns-3}:
\begin{itemize}[topsep=0pt, itemsep=0pt]
  \item \texttt{BulkSendApplication}. This traffic generator basically sends data as quickly as possible
  until MaxBytes is reached or the application is terminated (if MaxBytes is zero)
  \item \texttt{OnOffApplication}. After \texttt{StartApplication} is called, this traffic generator 
  alternates between "On" and "Off" states.
  \item \texttt{PacketSink}. This application is for receiving packets from any other applications, for
  example, from \texttt{BulkSendApplication} or \texttt{OnOffApplication}.
\end{itemize}

\subsection{Buildings Module}
The Buildings module provide various models, but the following are the most relevant for this thesis:

\subsubsection{Building class}

The \texttt{Building} model implements and tries to simulate real-life buildings, which affects
wireless communications in different ways.

A \texttt{Building} can be residential, office or commertial and has different types of external
wall (wood, concrete with/out windows, stone blocks), has a number of floors and rooms in each floor.

Some limitations have to be made:
\begin{itemize}[topsep=0pt]
  \item A \texttt{Building} is represented as a rectangular parallelepiped.
  \item The walls needs to be parallel to the cardinal coordinates.
  \item A \texttt{Building} is a grid of rooms, with z axis as the floor number and the x and y room
  indexes start from 1 and increses along the x and y axis.
  \item All the rooms are of the same size.
\end{itemize}

\subsubsection{MobilityBuildingInfo class}

The \texttt{MobilityBuildingInfo} keeps track of the mobility and positional information of 
the nodes with respect to buildings in a simulation. A node can be inside or outside of
a building and if the node is indoors, this class knows in which building and in which room
the node is positioned.

\subsubsection{ItuR1238PropagationLossModel}

This class provides an ITU P.1238-based building-dependent indoor propagation loss model 
that includes losses owing to building type (i.e., residential, office and commercial). 
The following is the analytical expression:

\begin{equation}
  L_{total} = 20\,log\,f + N\,log\,d + L_f (n) - 28[dB]
\end{equation}

where ${N}$ is the power loss coefficient, ${L_f}$ is the loss depending of type of building,
${f}$ is the frequency [MHz] and ${d}$ is de distance [m].

\subsubsection{BuildingPropagationLossModel}

The \texttt{BuildingsPropagationLossModel} adds a set of pathloss model elements that are 
building-dependent and can be used to design various pathloss logics. The elements of the 
pathloss model are discussed in the subsections below.

\begin{itemize}
  \item \textbf{External Wall Loss}

  This component simulates the loss of communication from indoors to outdoors and vice versa through walls.
  \item \textbf{Internal Wall Loss}

  This component simulates the loss of penetration in indoor-to-indoor communications within a single building.
  \item \textbf{Height Gain Model}

  This component simulates the gain caused by the transmitting equipment being on a higher floor than the ground.
  \item \textbf{Shadowing Model}

  The shadowing is represented using a log-normal distribution with a variable standard deviation
  as a function of the \texttt{MobilityModel} instances' relative position (inside or outdoor). For each 
  pair of \texttt{MobilityModels}, a single random value is generated and remains constant during the 
  simulation. As a result, the model is only suitable for static nodes.
\end{itemize}

\subsubsection{Pathloss logics}
The pathloss logic provided by inheriting from \texttt{BuildingsPropagationLossModel}
is described in the following sections.
\begin{itemize}
  \item \textbf{HybridBuildingsPropagationLossModel}

  In order to imitate multiple outdoor and interior circumstances, as well as indoor-to-outdoor 
  and outdoor-to-indoor scenarios, the \texttt{HybridBuildingsPropagationLossModel} was created by 
  combining various well-known pathloss models. In particular, this class combines the pathloss 
  models listed below:

  \item \textbf{OhBuildingPropagationLossModel}
  This is a simpler propagation loss model. It uses the \texttt{OkumuraHataPropagationLossModel} and 
  also taking account the pathloss elements of the \texttt{BuildingPropagationLossModel}.
\end{itemize}

\subsection{Internet Module}
This module includes the implementations of TCP/IP related components like IPv4, ARP, UDP, TCP and so on.
A \texttt{Node} with the Internet Stack installed is called a Internet Node.

In order to use the Internet Protocol, a node should be assigned an IP address. It can be done manually
or through the \textit{Dynamic Host Configuration Protocol (DHCP)}.

Full bidirectional TCP with connection setup and close logic is supported by the native \textit{ns-3} TCP model.
Various TCP congestion algorithms are also available, such as New Reno, Cubic, HighSpeed, etc.
The \textit{TCP} congestion algortihms affects on the adaptation algorithms, but it will
not be the focus on this thesis.


\subsection{Mobility Module}
The mobility module in \textit{ns-3} includes model to keep track the position and movement of the nodes and objects 
in cartesian coordinates and also a number of helper classes used for placing nodes and set up mobility 
models.

The \texttt{MobilityModel} is the base class for all the subclasses for different moving paths or behaviours.
The class \texttt{PositionAllocator} is typically used for seting the initial position of objects. \texttt{MobilityHelper}
combines a mobility model and position allocator used for adding mobility capabilities for a set 
of nodes.

Some useful subclasses of \texttt{MobilityModel} are:
\begin{itemize}[noitemsep, topsep=0pt]
  \item \texttt{ConstantPositionMobilityModel} for stationary nodes.
  \item \texttt{ConstantVelocityMobilityModel} for contant velocity moving nodes.
  \item \texttt{RandomWalk2SMobilityModel} for random walking in a 2D plane.
\end{itemize}


\subsection{Network Module}
The Network Module includes implementations of network related classes like 
\texttt{Packet}, \texttt{NetDevice}, TCP and UDP Sockets, etc.

\subsubsection{Packets}
A network packet is compesed by a byte buffer, a group of tags and metadata.
The serialized content of the headers and trailers added to a packet is stored in the byte buffer. 
The serialized form of these headers is expected to match the serialized representation of real 
network packets bit for bit, implying that the content of a packet buffer is supposed to be the 
same as that of a real packet.

\subsubsection{Sockets}
A socket is an abstraction that enables applications to communicate with other Internet hosts, among other services,
and exchange reliable byte streams and unreliable datagrams.

\begin{itemize}
  \item \textbf{ns-3 Sockets API} 
  
  The native sockets API for \textit{ns-3} provides an interface 
  to TCP and UDP. Although, the \texttt{ns3::Socket} have some differences compared to
  real sockets. 

  \item \textbf{Using Sockets} 

  In \textit{ns-3}, if an application wants to use sockets must create one first. By 
  calling \texttt{CreateSocket}, \textit{ns-3} creates a smart pointer to a \texttt{Socket} object.
  \textit{ns-3} sockets have all the functions of a real socket, including bind, connect, send, receive
  and close.

  In addition, the \texttt{Socket} class can set up events to make callbacks. For example, 
  \texttt{SetConnectCallback} is called when a connection is made, whether it succeeded of 
  failed, \texttt{SetSendCallback} is invoked when the send buffer is available
  and \texttt{SetRecvCallback} will notify when the data is received.
\end{itemize}

\subsection{PointToPoint NetDevice}
The \textit{ns-3} point-to-point model simulates a very basic point-to-point data link that connects 
two \texttt{PointToPointNetDevice} devices across a \texttt{PointToPointChannel}. This can be compared to a 
full duplex RS-232 or RS-422 connection with no handshaking and no null modem.

The create point-to-point net devices and channels, \texttt{PointToPointHelper} is used. To
connect two nodes, simply call the \texttt{Install} function.

\subsection{LTE Module}
There are two main components in the LTE-EPC simulation model.

\begin{itemize}[topsep=0pt, noitemsep]
  \item \textbf{LTE Model}. Includes models for the UE and the eNodeB nodes. Also the LTE Radio Protocol
  Stack (PHY, MAC, RLC, etc.).

  \item \textbf{EPC Model}. Includes models for the entities, interfaces and protocols in the Evolved Packet Core.
\end{itemize}

\subsubsection{LteHelper}
The \texttt{LteHelper} is a helper which manages the LTE radio access network's 
configuration as well as the setup and release of EPS bearers. The API definition and 
implementation are both provided by the \texttt{LteHelper} class.

A code snipper to create UEs and eNodeBs with \texttt{LteHelper} is found in \autoref{sec:ltehelpercode}

\begin{itemize}
  \item \textbf{Network Attachment}
  
  To connect an UE to the network, the UE needs to be attached to an eNodeB. This is done by calling the
  \texttt{LteHelper::Attach} function. The are two possible ways for network attachment.

  \begin{itemize}
    \item[$\circ$] \textbf{Automatic Attachment}

    This method uses the strengh of the received signal as the criteria to choose, 
    in the initial cell selection process, which eNodeB to connect to.
    
    \item[$\circ$] \textbf{Manual Attachment}
    Alternatively, selecting the eNodeB at the beginning of the simulation is also possible.

  \end{itemize}

  \item \textbf{Simulation Output}
  
  The LTE module offer PHY, MAC, RLC, and PDCP level \textit{Key Performance Indicators (KPI)}
  that can be enabled using \textbf{LteHelper}:

  \begin{lstlisting}[language=myC++,caption={Enable LTE trace outputs}, captionpos=b]
    lteHelper->EnablePhyTraces ();
    lteHelper->EnableMacTraces ();
    lteHelper->EnableRlcTraces ();
    lteHelper->EnablePdcpTraces ();
  \end{lstlisting}
\end{itemize}

\subsubsection{EpcHelper}
The \texttt{EpcHelper} allows the simulation of the Evolve Packet Core. The usage of EPC with 
LTE devices allows for IPv4 and IPv6 networking. To put it another way,it is possible to use 
standard \textit{ns-3} apps and sockets across IPv4 and IPv6 via LTE, as well as connect an LTE network 
to any other IPv4 and IPv6 network in the simulation.

It is possible to access the SGW and PGW nodes by calling the \texttt{GetSgwNode} and the 
\texttt{GetPgwNode} respectively.

\subsubsection{MAC}
\begin{itemize}
  \item \textbf{MAC Scheduler}
  
  In \textit{ns-3}, there are several types of MAC schedulers available. User can choose which 
  one to use with the \texttt{LteHelper}:

  \begin{itemize}[noitemsep, topsep=0pt]
    \item FD-MT (Frequency Domain Maximum Throughput Scheduler) 
    \item TD-MT (Time Domain Maximum Throughput Scheduler) 
    \item TTA (Throughput to Average Scheduler) 
    \item FD-BET (Frequency Domain Blind Average Throughput Scheduler) 
    \item TD-BET (Time Domain Blind Average Throughput Scheduler) 
    \item FD-TBFQ (Frequency Domain Token Bank Fair Queue Scheduler) 
    \item TD-TBFQ (Time Domain Token Bank Fair Queue Scheduler) 
    \item PSS (Priority Set Scheduler Scheduler) 
  \end{itemize}



  \item \textbf{AMC \& CQI}
  
  In terms of selecting MCSs and generating CQIs, the simulator offers two options. 
  The first is the implementation by \cite{piro} and operates on a per-RB basis, and the other 
  is based on the physical error mode.

\end{itemize}

% \subsection{UE PHY Measurements Model}

\subsubsection{Mobility Model with Buildings}
The propagation model to be used with the LTE module is defined in the
Buildings module.

This creates a residential building, concrete with windows, with 
3 floors and 6 rooms each floor. It is set that all UEs are in a 
constant position, but other mobility models are also possible.

The LTE module is also compatible with existing propagation loss models.
Only the propagation from the UEs to the base stations are computed.

\subsubsection{AntennaModel \& MIMO Model}
Any model of \texttt{AntennaModel} is supported, by default, the \texttt{IsotropicAntennaModel} is
used for both eNBs and UEs. In case of using multiple antennas, \textit{ns-3} offers different MIMO operation 
modes.

\subsubsection{Radio Environment Maps}

With this class is possible to create a Radio Environment Map (REM), 
which is a uniform 2D grid of values that reflect the signal-to-noise ratio in the
downlink with regard to the eNB with the strongest signal at each point, to a file by 
using the class \texttt{RadioEnvironmentMapHelper}.

Using a software like gnuplot\footnote{http://www.gnuplot.info/}, the output file can be visualized.

\autoref{fig:rem} shows an example of a Radio Environment Map.

\begin{figure}[h]
  \centering
  \includegraphics[width=0.77\textwidth]{img/rem.png}
  \caption{Example Radio Environment Map. Source:\cite{ns3} }
  \label{fig:rem}
\end{figure}

\section{Parameter Configuration}
The \textit{ns-3} attribute system is the entity that manages all the parameters. It is 
possible to use input files using \texttt{ConfigStore} and set initial values for default 
and global paramaters.

It is important to include \texttt{\#include "ns3/config-store.h"} in the script. Then create
a text file named as defined before and specify the new default values to be used.