\chapter{Impact} \label{chap:impact}


\section{Socio-economic \& Ambiental Impact} \label{sec:social_impact}

Video traffic continues to grow rapidly, both in terms of overall volume 
and as a percentage of total traffic, especially during this COVID-19
global pandemic. 
Improvements in technology, such as 
4K, 8K video, videoconferences or video-game streaming, and the ever-present 
availability of consumable media, are all contributing to this growth.

The storage and bandwidth costs of transmitting video to these 
increases as streaming services scale up to fulfill the demand for more 
content across more devices. The cost of storing and streaming video may 
be dramatically reduced by 
efficiently providing high-quality video at scale to a wide range of devices, 
while also enhancing playback quality for consumers.

In future years, 5G's speed, capacity, and connectivity will open up a slew of 
possibilities for environmental protection and preservation. Energy efficiency, 
greenhouse gas emissions, and the utilization of renewable energy will all benefit from 
5G technology. Even though 5G technologies are more efficient 
compared to 4G, it is estimated that mobile data will almost fourfold\cite{gsm1} and 
it can be very challenging to reduce the pollution the mobile networks
can produce.